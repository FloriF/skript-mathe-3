\documentclass[12pt,titlepage, pdf]{article}

%Packages%

\usepackage[ngerman]{babel}
\usepackage[utf8]{inputenc}
\usepackage{amssymb}
\usepackage{amsmath}
\usepackage{dsfont}
\usepackage{mathtools}
\usepackage[pd,col,3d]{mgtex_24}
\usepackage{tikz}
\usepackage{marvosym}
\usepackage{ mathrsfs }
\usepackage{gensymb}
\usepackage{framed}
\usepackage{arydshln}

%TitlePage%

\begin{titlepage}
	\title{Mathematik III}
	\date{15.11.2016}
\end{titlepage}

%NewCommands%

\newcommand{\R}{\mathds{R}}
\newcommand{\N}{\mathds{N}}
\newcommand{\uline}[1]{\underline{#1}}
\newcommand{\id}{\textrm{id}}
\newcount\colveccount
\newcommand*\colvec[1]{
	\global\colveccount#1
	\begin{pmatrix}
		\colvecnext
	}
	\def\colvecnext#1{
		#1
		\global\advance\colveccount-1
		\ifnum\colveccount>0
		\\
		\expandafter\colvecnext
		\else
	\end{pmatrix}
	\fi
}
\newcommand{\vecspace}[2]{\langle#1\rangle_{#2}}
\newcommand{\vecspaceR}[1]{\vecspace{#1}{\R}}

%ReNewCommands%

\renewcommand{\>}{\rightarrow}
\renewcommand{\*}{\cdot}
\renewcommand{\O}{\mathcal{O}}
\renewcommand{\phi}{\varphi}
\renewenvironment{rcases}{% 
	\left.\renewcommand*\lbrace.% 
	\begin{cases}}% 
	{\end{cases}\right\rbrace}
\renewcommand{\vec}[1]{\colvec{#1}}

%Document%

\begin{document}
	\maketitle
	\tableofcontents
	\newpage
	\section{Vektorräume}
	\small{Bemerkung: 1.1-1.10 identisch mit 8.1-8.10 aus Mathematik 2, SS16}
	\subsection{Definition (Reelle Vektorräume)}
	Ein \underline{$\R$-Vektorraum V} ist eine nichtleere Menge, deren Elemente \underline{Vektoren} genannt werden (Bezeichnung mittels kleiner lateinischer Buchstaben, $v,w,x,y,...$), auf der eine Addition $+$ definiert ist, $+\colon V\times V\>V$; und eine Multiplikation mit reellen Zahlen ('Skalare') (Bezeichnung mittels kleiner griechischer Buchstaben $\alpha, \beta, \gamma, \lambda,\mu,...$), $\*\colon\R\times V\>V$, so dass gilt:
	\begin{itemize}
		\item[(1.1)] $u+v+w=u+(v+w)\qquad\forall u,v,w\in V$
	   	\item[(1.2)] Es existiert ein Vektor $\O\in V$ ('\underline{Nullvektor}') mit $v+\O=\O+v=v\qquad\forall v\in V$
	   	\item[(1.3)] Zu jedem $v\in V$ existiert ein Vektor $-v\in V$ mit $v+(-v)=\O$
		\item[(1.4)] $u+v=v+u\qquad\forall u,v\in V$
	\end{itemize}
	(Diese Eigenschaften (1.1) bis (1.4) kann man zusammenfassen als '$(V,+)$ ist eine kommutative Gruppe').
	\begin{itemize}
		\item[(2.1)] $\overset{\textrm{Addition in }\R}{(\lambda+\mu)}\*v=\lambda\*v\overset{\textrm{Addition in }V}{+}\mu\*v\qquad\forall\lambda,\mu\in\R,v\in V$
		\item[(2.2)] $\lambda(v+w)=\lambda v+\lambda w\qquad\forall\lambda\in\R,v,w\in V$
		\item[(2.3)] $\overset{\textrm{Multiplikation in }\R}{(\lambda\*\mu)}\*v=\lambda\*\overset{\textrm{Multiplikation mit Skalar}}{(\mu\*v)}\qquad\forall\lambda,\mu\in\R,v\in V$
		\item[(2.4)] $1\*v=v\qquad\forall v\in V$
	\end{itemize}
	\subsection{Beispiel}
	\begin{itemize}
		\item[a)] trivialer Vektorraum Nullraum: $V=\{\O\}$\\
		Es gilt $\O+\O\coloneqq\O,\quad\lambda\*\O\coloneqq\O\qquad\forall\lambda\in\R$
		\item[b)] $V=\R^n$, Raum aller 'Spaltenvektoren' der Länge $n$ über $\R$, Elemente haben die Form $\vec3{x_1}{...}{x_n}$ mit $x_1,...,x_n\in\R$.\\
		$\O=\vec3{0}{...}{0},\quad\vec3{x_1}{...}{x_n}+\vec3{y_1}{...}{y_n}=\vec3{x_1+y_1}{...}{x_n+y_n},\quad\lambda\*\vec3{x_1}{...}{x_n}=\vec3{\lambda\*x_1}{...}{\lambda\*x_n}$
		\item[c)] $\R$ ist ein $\R$-Vektorraum.\\
		Vektoren: reelle Zahlen.\\
		Skalare: reelle Zahlen.\\
		$\O=0$
		\item[d)] Funktionenraum:\\
		$M\neq\emptyset$ Menge. $V=\mathcal{F}(M,\R)\coloneqq\{f\colon M\>\R\}$\\
		Menge der auf $M$ definierten reellen Funktionen.\\
		Für $f,g\in V,\quad\lambda\in\R$ sei
		\begin{itemize}
			\item $f+g\colon M\>\R,\quad(f+g)(x)=f(x)+g(x)\quad\forall x\in M$
			\item $\lambda\*f\colon M\>\R,\quad(\lambda\*f)(x)=\lambda\*f(x)\quad\forall x\in M$
		\end{itemize}
		Dann ist $V$ mit $\R,+,\*$ ein Vektorraum. Nullvektor ist $f=0\colon M\>\R,\quad f(x)=0\qquad\forall x\in M$.\\
		(kurz: $f\equiv0$, identisch Null)
	\end{itemize}
	\subsection{Lemma}
	Sei $V$ ein $\R$-Vektorraum, $v\in V,\quad\lambda\in\R$
	\begin{itemize}
		\item[a)] $0\*v=\O$
		\item[b)] $\lambda\*\O=\O$
		\item[c)] Zu jedem $v\in V$ ist der Vektor $-v$ aus (1.3) in 8.1 eindeutig bestimmt.
		\item[d)] $(-1)\*v=-v$
	\end{itemize}
	\subsubsection*{Beweis}
	\begin{itemize}
		\item[a)] \begin{align*}
		\O\overset{\textrm{(1.3)}}{=}\overbrace{\underbracket{0\*v}_{}}^{x}+\overbrace{(-0\*v)}^{-x}&=\underbracket{(0+0)v}_{}+(-0\*v)\\
		&\overset{\textrm{(2.1)}}{=}(0\*v+0\*v)+(-0\*v)\\
		&\overset{\textrm{(1.1)}}{=}0\*v+(0*v+(-0\*v))\\
		&\overset{\textrm{(1.3)}}{=}0\*v+\O\\
		&\overset{\textrm{(1.2)}}{=}0\*v
		\end{align*}
		\item[b)] Wie a), starte mit $\O=\lambda\*\O+(-\lambda\*\O)$, erhalte $\O=\lambda\*\O$
		\item[d)] \begin{align*}
		\underbracket{v+(-1\*v)}_{}&=1\*v+(-1\*v)\\
		&\overset{\textrm{(2.1)}}{=}(1+(-1))v\\
		&=0\*v\\
		&\overset{\textrm{a)}}{=}\O\\
		&\overset{\textrm{(1.3)}}{=}v+(-v)
		\end{align*}
		Addiere auf beiden Seiten $-v$:
		\begin{align*}
		\underbracket{v+(-1)v}_{}+(-v)&=v+(-v)+(-v)\\
		&\Rightarrow-1\*v=-v		
		\end{align*}
		\item[c)] Angenommen, zu $v\in V$ gibt es $-v$ und $-v^\prime$ mit $v+(-v)=\O$ und $v+(-v^\prime)=\O$. Dann ist $v+(-v)=v+(-v^\prime)\overset{+(-v)\textrm{auf beiden Seiten}}{\Rightarrow}-v=-v^\prime$
	\end{itemize}
	\hfill$\square$
	\subsection{Definition (Untervektorraum)}
	Sei $V$ ein $\R$-Vektorraum.\\
	Eine Teilmenge $U\subseteq V,\quad U\neq\emptyset$ heißt \underline{Unter(vektor)raum von $V$}, falls $U$ bezüglich der Addition auf $V$ und der Multiplikation mit Skalaren selbst ein Vektorraum ist.
	\subsection{Beispiel}
	\begin{itemize}
		\item[a)] $V=\R^2,\quad U=\{\vec2{0}{0}\}$ ist Unterraum von $V$
		\item[b)] $V=\R^2,\quad U=\{\vec2{1}{2}\}$ ist kein Unterraum von $V$, z.B. (1.2) ist verletzt, Addition funktioniert auch nicht: $\vec2{1}{2}+\vec2{1}{2}=\vec2{2}{4}\notin U$
		\item[c)] $V=\R^2,\quad U=\{\vec2{\lambda}{0}|\lambda\in\R\}$ ist ein Unterraum von $V$ (prüfe alle Eigenschaften von Definition 8.1) $\>$ umständlich, einfacher geht es mit 8.6
	\end{itemize}
	\subsection{Satz (Unterraumkriterium)}
	Sei $V$ ein $\R$-Vektorraum, sei $\emptyset\neq U\subseteq V$.\\
	Dann ist $U$ Unterraum von $V$ genau dann, wenn gilt ($\Leftrightarrow$):
	\begin{itemize}
		\item[(1)] $v\in U,\quad\lambda\in\R\Rightarrow\lambda\*v\in U$
		\item[(2)] $v,w\in U\Rightarrow v+w\in U$
	\end{itemize}
	(oder äquivalent: $\forall v,w\in U, \forall\lambda,\mu\in\R$ ist $\lambda\*v+\mu\*w\in U$)\\
	Man sagt: $U$ ist abgeschlossen bezüglich der Vektoraddition und der Multiplikation mit Skalaren.
	\subsubsection*{Beweis}
	\begin{itemize}
		\item[$\Rightarrow$] ist klar, da $U$ laut Definition 8.4 selbst Vektorraum
		\item[$\Leftarrow$] rechne die Vektorraumaxiome nach (Definition 8.1, also z.B. $\O\in U$,...)
	\end{itemize}
	\hfill$\square$
	\subsection{Beispiel}
	\begin{itemize}
		\item[a)] $\quad$\\
		\begin{minipage}[c]{0.5\textwidth}
			$V$ ist ein $\R$-Vektorraum, $\O\neq v\in V$.\\
			Dann ist $G=\{\lambda\*v|\lambda\in\R\}$ ein Unterraum.\\
			$V=\R^2,\R^3$: $G$ ist Gerade durch Nullpunkt (geometrisch), z.B.\\ $v=\vec2{2}{1},w=\vec2{1}{2}$\\
			Aber: $G^\prime=\{w+\lambda\*v|\lambda\in\R,\quad w\in V\}$ ist kein Unterraum für $w\neq \mu\*v,\quad \mu\in\R$. Warum? Z.B. $\O\notin G^\prime$
		\end{minipage}
		\begin{minipage}[c]{0.25\textwidth}$\quad$
			\begin{tikzpicture}[scale=0.5]
			\draw[->]
			(-0.2,0) -- (5,0) node[right] {$x$};
			\draw[->]
			(0,-0.5) -- (0,5) node[above] {$f(x)$};
			\draw[color=blue, thick, ->]
			(0,0)--(2,1)node[below]{$v$};
			\draw[color=blue, thick, ->]
			(0,0)--(1,2)node[above]{$w$};
			\draw[color=blue, ->]
			(0,0)--(4,2)node[below]{$\quad2\*v$, $G$};
			\draw[color=blue, ->]
			(0,0)--(2,4)node[above]{$2\*w$};
			\draw[color=blue, ->]
			(0,1.5)--(3,3)node[right]{$G^\prime$};
			\end{tikzpicture}
		\end{minipage}
		\item[b)] $V=\R^3,\qquad U_1=\{\vec3{x_1}{x_2}{x_3}\in\R^3|x_1+x_2-x_3=0\}$ ist Unterraum. Wir zeigen (1), (2) aus 8.6:
		\begin{itemize}
			\item $U_1\neq\emptyset$, z.B. $\O=\vec3{0}{0}{0}\in U_1$, denn $\overset{x_1}{0}+\overset{x_2}{0}-\overset{x_3}{0}=0$
			\item[(1)] Sei $\lambda\in\R,\quad v=\vec3{v_1}{v_2}{v_3}\in U_1$, d.h. $v_1+v_2-v_3=0$\\
			Prüfe: Ist $\lambda\*v\in U_1$? $\lambda\*v=\vec3{\lambda\*v_1}{\lambda\*v_2}{\lambda\*v_3}$
			\begin{align*}
			\lambda\*v_1+\lambda\*v_2-\lambda\*v_3&=\lambda(v_1+v_2-v_3)\\
			&=\lambda\*0\\
			&=0
			\end{align*}
			Also ist $\lambda\*v\in U_1$
			\item[(2)] Seien $v=\vec3{v_1}{v_2}{v_3},\quad w=\vec3{w_1}{w_2}{w_3}\in U_1$, d.h. $v_1+v_2-v_3=0,\quad w_1+w_2-w_3=0$. Gilt $v+w\in U_1$?  $v+w=\vec3{v_1+w_1}{v_2+w_2}{v_3+w_3}$
			\begin{align*}
			(v_1+w_1)+(v_2+w_2)-(v_3+w_3)&=\underbrace{(v_1+v_2-v_3)}_{=0}+\underbrace{(w_1+w_2-w_3)}_{=0}\\
			&=0
			\end{align*}
			Also $v+w\in U_1$
			\item Geometrische Interpretation:\\
			\begin{align*}
			U_1&=\{\vec3{x_1}{x_2}{x_1+x_2}|x_1,\quad x_2\in\R\}\\
			&=\{x_1\*\vec3{1}{0}{1}+x_2\*\vec3{0}{1}{1}|x_1,\quad x_2\in\R\}
			\end{align*}
			D.h. $U_1$ ist die Ebene durch $O=\vec3{0}{0}{0}$ mit den Richtungsvektoren $\vec3{1}{0}{1}$ und $\vec3{0}{1}{1}$
		\end{itemize}
		\item[c)] $U_2=\{\vec3{x_1}{x_2}{x_3}\in\R^3|x_1+x_2-x_3=1\}$ ist kein Unterraum. Z.B. $\vec3{0}{0}{0}=\O\notin U_2$: $0+0-0=0\neq1$.\\
		Anderes Argument: Sei $\lambda\in\R,\quad x=\vec3{x_1}{x_2}{x_3}\in U_2$, d.h. $x_1+x_2-x_3=1$. Gilt $\lambda\*x\in U_2$? $\lambda\*x=\vec3{\lambda x_1}{\lambda x_2}{\lambda x_3}$
		\begin{align*}
		\lambda x_1+\lambda x_2-\lambda x_3&=\lambda\underbrace{(x_1+x_2-x_3)}_{=1}\\
		&=\underbrace{\lambda=1}_{\textrm{nur für }\lambda=1}
		\end{align*}
		$\Rightarrow$ nicht erfüllt für $\lambda\neq1$.\\
		Geometrische Interpretation:\\
		\begin{align*}
		U_2&=\{\vec3{x_1}{x_2}{x_1+x_2-1}|x_1,\quad x_2\in\R\}\\
		&=\{\vec3{0}{0}{-1}+x_1\*\vec3{1}{0}{1}+x_2\*\vec3{0}{1}{1}|x_1,\quad x_2\in\R\}
		\end{align*}
		Ebene durch $\vec3{0}{0}{-1}$ mit Richtungsvektoren $\vec3{1}{0}{1}$ und $\vec3{0}{1}{1}$
		\item[d)] $U_3=\{\vec3{x_1}{x_2}{x_3}\in\R^3|x_1^2+x_2^2+x_3^2\leq1\}$ ist kein Unterraum, z.B.\\
		$\vec3{1}{0}{0}\in U_3,\qquad1^2+0^2+^2\leq1\quad\checkmark$, aber\\
		$2\*\vec3{1}{0}{0}=\vec3{2}{0}{0}\notin U_3$, denn $2^2+0^2+0^2\nleq1$\\
		Geometrische Interpretation:\\
		$U_3$ ist eine Kugel um $\vec3{0}{0}{0}$ mit Radius 1
		\item[e)] $I\subseteq\R$ Intervall\\Menge $C(I)$ ($C$: continuous, stetig) der stetigen Funktionen auf $I$ ist Unterraum von $\mathcal{F}(I,\R)$ (vgl. Beispiel 8.2d)).\\
		Menge der diffbaren Funktionen auf $I$ ist Unterraum von $C(I)$.
	\end{itemize}
	\subsection{Satz}
	$V$ ist ein $\R$.Vektorraum, $U_1,U_2$ sind Unterräume von $V$.
	\begin{itemize}
		\item[a)] $U_1\cap U_2=\{u\in V|u\in U_1\wedge u\in U_2\}$ ist Unterraum von $V$.
		\item[b)] $U_1+U_2\coloneqq\{u_1+u_2|u_1\in U_1\wedge u_2\in U_2\}$ \underline{Summe} von $U_1,U_2$ ist Unterraum von $V$\\
		(das ist nicht die Vereinigung $U_1\cap U_2$!)
	\end{itemize}
	\subsubsection*{Beweis}
	Prüfe Unterraumkriterium 8.6
	\begin{itemize}
		\item[a)] Übung: Prüfe $\O\in U_1\cap U_2$? $\checkmark$, (1), (2)
		\item[b)] \begin{itemize}
			\item $U_1+U_2\neq\emptyset$, denn $U_1+U_2\ni\O=\underbrace{\O}_{\in U_1}+\underbrace{\O}_{\in U_2}$
			\item Seien $v=u_1+u_2, \quad u_1\in U_1,\quad u_2\in U_2$ und\\
			$w=u_1^\prime+u_2^\prime,\quad u_1^\prime\in U_1,\quad u_2^\prime\in U_2$,\\
			also $v,w\in U_1+U_2$ und $\lambda,\mu\in\R$.\\
			\begin{align*}
			\Rightarrow\qquad\lambda v+\mu v&=\lambda(u_1+u_2)+\mu(u_1^\prime+u_2^\prime)\\
			&=\underbrace{\lambda u_1+\mu u_1^\prime}_{\in U_1}+\underbrace{\lambda u_2+\mu u_2^\prime}_{\in U_2}
			&\in U_1+U_2
			\end{align*}
		\end{itemize}
	\end{itemize}
	\subsection{Bemerkung}
	\begin{itemize}
		\item[a)] lässt sich für unendlich viele Unterräume ausweiten
		\item[b)] lässt sich für endlich viele Unterräume ausweiten
		\item $U_1\cup U_2$ ist im Allgemeinen \underline{kein} Unterraum
	\end{itemize}
	\subsection{Beispiel}
	\begin{itemize}
		\item $v=\vec2{1}{0}\in\R^2\qquad\qquad G_1=\{\lambda v|\lambda\in\R\}$
		\item $w=\vec2{2}{1}\in\R^2\qquad\qquad G_2=\{\mu w|\mu\in\R\}$
	\end{itemize}
	(vgl. 8.7a), Geraden durch $\vec2{0}{0}$, Unterräume
	\begin{itemize}
		\item $G_1+G_2$ ist Ebene
		\item $G_1\cap G_2$ ist $\O=\vec2{0}{0}$
	\end{itemize}

%MATHEMATIK 3%%MATHEMATIK 3%%MATHEMATIK 3%%MATHEMATIK 3%
%MATHEMATIK 3%%MATHEMATIK 3%%MATHEMATIK 3%%MATHEMATIK 3%
%MATHEMATIK 3%%MATHEMATIK 3%%MATHEMATIK 3%%MATHEMATIK 3%

	\subsection{Beispiel}
	\marginpar{18.10.16}
	\begin{minipage}[c]{0.5\textwidth}
	\InitGraph{5}{8}{1}{1}{0.7cm}
	\Viewpoint(300,70,10,8)[1]
	\SetCMYKColor(0.5,0,0,0)
	\ShowFullPlaneThrough(2.1,1,1)NormalTo(0,2,-2)(2.1)
	\Text[t]{$E$}
	\SetCMYKColor(0,0,0,255)
	\DDArrowAt(0,0,0)(6,0,0)
	\Text[b]{x}
	\DDArrowAt(0,0,0)(4,1,1)
	\SetDashed
	\DDArrowAt(0,0,0)(0,1000,0)
	\Text[b]{y}
	\SetNormal
	\DDArrowAt(0,30,0)(0,1000,0)
	\DDArrowAt(0,0,0)(0,0,6)
	\Text[l]{z}
	\SetCMYKColor(0,1,1,0)
	\DDArrowAt(0,0,0)(2,0,0)
	\Text[b]{$v$}
	\SetCMYKColor(1,1,0,0)
	\DDArrowAt(0,0,0)(0,1,1)
	\Text[t]{$u$}
	\Text[t]{$u+2v$}
	\CloseGraph
	\end{minipage}
	\begin{minipage}[c]{0.5\textwidth}
		\begin{itemize}
			\item $ u = \vec3{0}{1}{1}$
			\item $ v = \vec3{2}{0}{0}$
			\item $ E = \{ \lambda_1 \cdot \vec3{0}{1}{1} + \lambda_2 \cdot \vec3{2}{0}{0} | \lambda_1, \lambda_2 \in \R\}$
		\end{itemize}
	\end{minipage}
\\
\\
\\
\begin{itemize}
	\item E $\subseteq \R^3 $ ist Untervektorraum (UVR) und wird \uline{aufgespannt/erzeugt} von $u$ und $v$. Man nennt $\{\vec3{0}{1}{1},\vec3{2}{0}{0}\}$ \uline{Erzeugendensystem} von $E$.
	\item D.h. $w \in E \Leftrightarrow \exists \lambda_1, \lambda_2 \in \R: w = \underbrace{\lambda_1 \cdot u + \lambda_2 \cdot v}_{\text{Linearkombination von $u$ und $v$}}$
	\item $w \notin E$, z.B. $w = \vec3{0}{0}{1}$ ergibt: \\
	$\vec3{0}{0}{1} = \lambda_1 \cdot u + \lambda_2 \cdot v = \lambda_1 \vec3{0}{1}{1} + \lambda_2 \vec3{2}{0}{0} \\
	\\
	 \Rightarrow \begin{rcases}
	 	\textrm{Letzte Zeile: }1=\lambda_1\\
	 	\textrm{Zweite Zeile: }0=\lambda_1
	 \end{rcases}$\Lightning
 \\
	$\Rightarrow \vec3{0}{0}{1} \notin E$
\end{itemize}
\subsection*{Fortsetzung Bsp. 1.11}
\marginpar{(Nachtrag vom 19.10.2016)}
\begin{itemize}
	\item[a)] $E = \langle \vec3{0}{1}{1}, \vec3{2}{0}{0} \rangle_{\R}$
	\item[b)] $\R^n$ wird erzeugt von $e_j = \vec5{0}{\vdots}{1}{\vdots}{0}$, wobei j die Stelle ist, an der der Vektor 1 ist. \\
	$R^n = \vecspaceR{\vec4{1}{0}{0}{\vdots}, \vec4{0}{1}{0}{\vdots}, ...,\vec4{0}{0}{\vdots}{1}}$ "kanonische Einheitsvektoren" \\
	$v = \vec3{v_1}{\vdots}{v_n} = v_1 \cdot e_1 + v_2 \cdot e_2 + ... + e_n \cdot v_n$
	\item[c)] Spannen $\vec2{1}{1}$ und $\vec2{1}{2}$ den $\R^2$ auf? \\
	Wenn ja, dann muss für $\vec2{x}{y} \in  \R^2\qquad \alpha, \beta \in \R$ existieren mit 
	\begin{align*}
	&&\alpha \cdot \vec2{1}{1} + \beta \cdot \vec2{1}{2} &= \vec2{x}{y}\\
	\Leftrightarrow&& \alpha + \beta &= x \\
	&&\alpha + 2\beta &= y \\
	\Rightarrow &&\alpha &= x - \beta\\
	&& & = y - 2 \beta \\
	\Leftrightarrow &&\beta &= y - x\\
	&&\alpha &= 2x -y 
	\end{align*}
	$\Rightarrow$\quad Allg. $ \vec2{x}{y} = (2x-y) \cdot \vec2{1}{1} + (y-x)\cdot \vec2{1}{2} \Rightarrow \R^2 = \langle \vec2{1}{1}, \vec2{1}{2} \rangle_{\R}$
	\item[d)] Spannen $\vec2{1}{2}$ und $\vec2{3}{6}$ den $\R^2$ auf? \\
	Nein, denn $\vec2{3}{6}$ ist $3 \cdot \vec2{1}{2} \Rightarrow \langle\vec2{1}{2}, \vec2{3}{6} \rangle_{\R} =  \langle \vec2{1}{2} \rangle_{\R} 
	= \{\lambda \cdot \vec2{1}{2}\vert \lambda \in \R \} \subsetneq \R^2$
	\item[e)]
	$\langle \vec2{0}{1}, \vec2{0}{1} \rangle_{\R} = \langle \vec2{1}{1}, \vec2{1}{2} \rangle_{\R} = \R^2$, d.h. Erzeugendensysteme sind \uline{nicht} eindeutig!
	\item[f)] 
	$\langle \vec2{1}{1}, \vec2{1}{2}, \vec2{2}{3} \rangle_{\R} = \langle \vec2{1}{1}, \vec2{1}{2} \rangle_{\R}$, da $\vec2{2}{3} = 
	\vec2{1}{1} + \vec2{1}{2}$.\\
	D.h. $M = \{\vec2{1}{1}, \vec2{1}{2}, \vec2{2}{3} \}$ ist kein \uline{minimales} Erzeugendensystem des $\R^2$, denn $v \in M$ kann immer dargestellt werden als Linearkombination von Vektoren aus $M \setminus {v}$. \\
	Man sagt: $\vec2{1}{1}, \vec2{1}{2}, \vec2{2}{3}$ sind \uline{linear abhängig}.
\end{itemize}
\subsection{Definition (Linearkombination, Erzeugendensystem)}
$V: \R$-VR (V ist Vektorraum in den reellen Zahlen) \\
\begin{itemize}
	\item[(i)] $v_1, ... , v_m \in V$ und $\lambda_1,...,\lambda_m \in \R$\\ Der Vektor $\lambda_1 \* v_1 + ... + \lambda_m \* v_m$ heißt \uline{Linearkombination} von $v_1,...,v_m$.
	\item[(ii)] Sei $M \subseteq V$. Dann ist
	\begin{center}
		$\langle M \rangle_{\R} = \{ \sum_{k = 1}^{n} \lambda_k \cdot v_k \vert \lambda_k \in \R, v_k \in M, n \in \mathbb{N}\}$
	\end{center}
	der \underline{von $M$ aufgespannte/erzeugte UVR} von V \\
	\\
	Vereinbarung: $\langle \emptyset \rangle = \{0\}$\\
	Schreibweise: $M = \{v_1,...,v_m\}$\\
	\noindent\hspace*{22mm}$\langle M \rangle_{\R} = \langle v_1,..., v_m\rangle_{\R} $
	\item[(iii)]
	Ist $V = \langle M \rangle_{\R}$, so heißt $M$ ein \uline{Erzeugendensystem} von $V$. $V$ heißt \uline{endlich erzeugt}, falls es ein endliches Erzeugendensystem gibt.
\end{itemize}
\subsection{Bemerkung}
$M \subseteq V \Rightarrow \langle M \rangle_{\R}$ ist der kleinste UVR von $V$, der $M$ enthält.\\
\subsubsection*{Beweis}
\begin{itemize}
	\item $\langle M \rangle_{\R}$ ist UVR. erfüllt Kriterien von 1.6, daher klar: \\
	1.6 2) erfüllt. $u \in \langle M \rangle_{\R} \Rightarrow u = \lambda_1 \cdot v_1 + ... + \lambda_n \cdot v_n\quad(M = \{v_1, ..., v_n \})\\ \Rightarrow \lambda \cdot u = \underbrace{\lambda  \lambda_1}_{\in \R} \cdot v_1 + ... + \underbrace{\lambda \lambda_n}_{\in \R} \cdot v_n$\\
	1.6 3) ähnlich.
	\item Angenommen $U$ ist der kleinste UVR, so dass $M \subseteq U$. \\
	Z. z.: $\langle M \rangle_{\R} = U.$\\
	Wegen 1.6 enthält $U$ alle Linearkombinationen von Vektoren aus M. \\
	$\Rightarrow \langle M \rangle_{\R} \subseteq U \Rightarrow U$ kann nicht kleiner sein als $\langle M \rangle_{\R} \Rightarrow \langle M \rangle_{\R} = U$\hfill$\square$
\end{itemize}
\subsection*{Ergänzung zu 1.13}
\marginpar{19.10.16}
	Bsp: $M = \{\vec3{0}{1}{1} \} \Rightarrow \vecspaceR{M} = \{ \lambda \vec3{0}{1}{1} \vert  \lambda  \in  \R \}$ Gerade 
	\begin{itemize}
		\item $\vecspaceR{M} \supseteq M$
		\item $ E = \{\lambda_1 \vec3{0}{1}{1}  + \lambda_2 \vec3{2}{0}{0} \vert \lambda_1, \lambda_2 \in \R \} \supseteq M $
	\end{itemize}
	$\vecspaceR{M}$ Gerade, E Ebene, d.h. E ist größer als $\vecspaceR{M}$\\
	$\vecspaceR{M}$ ist der kleinste UVR von $\R^3$, der M enthält.
	\subsection{Definition (Lineare Unabhängigkeit)}	
	\begin{itemize}
		\item $V\colon\quad\R - VR,\quad v_1,...,v_n$ heißen \uline{linear unabhängig}, wenn gilt: 
		\begin{center}
			$\begin{rcases}
			\lambda_1 \cdot v_1 + ... + \lambda_m \cdot v_m = 0 \\
			\lambda_1,...,\lambda_m\in\R
			\end{rcases}
			\Rightarrow \underbrace{\lambda = \lambda_2 = ... = \lambda_m = 0}_{\textrm{einzige Lösung!}}$
		\end{center}
		\item $M \subseteq V$ heißt linear unabhängig, wenn gilt: \\
		Für beliebiges $m \in \N $ und $v_1,...,v_m \in M$ paarweise verschieden sind $v_1,...,v_m$ linear unabhängig
		\item Ist in obigen beiden Fällen (mindestens) $\lambda_i \neq 0$, dann sind die Vektoren linear abhängig
	\end{itemize}
\subsection{Beispiel}
\begin{itemize}
	\item[a)] $\mathcal{O}$ ist linear abhängig, da $\lambda \cdot \mathcal{O} = 0 \qquad \forall \lambda \neq 0$
	\item[b)] Sind $\vec2{1}{2} , \vec2{-3}{1}, \vec2{1}{-5}$ linear abhängig in $\R^2$ ? \\
	$\lambda_1 \cdot \vec2{1}{2} + \lambda_2 \cdot \vec2{-3}{1} + \lambda_3 \cdot \vec2{1}{-5} = \mathcal{O}$\\
	$\begin{cases}
	\text{I}  \qquad \lambda_1 -3\lambda_2 + \lambda_3  &= 0 \\
	\text{II} \qquad 2 \lambda_1 + \lambda_2 - 5 \lambda_3 &= 0
	\end{cases}$\quad
	Erfüllt für $\lambda_1 = \lambda_2 = \lambda_3 = 0$. Aber hier gibt es noch die Lösung: $\lambda_1 = 2,\quad \lambda_2 = \lambda_3 = 1$!\\
	$\Rightarrow$ Vektoren sind linear abhängig 
	\item[c)] 
	$\vec3{1}{0}{0}, \vec3{0}{1}{0}, \vec3{0}{0}{1}$ linear unabhängig (l.u.) in $\R^3$ 
	\item[d)]
	$v \neq \mathcal{O},\quad v \in V,\quad v$, ist linear unabhängig \\
	Angenommen es existiert $\lambda \neq 0$ mit $\lambda \cdot v = 0$. \\
	$\Rightarrow v = (\frac{1}{\lambda} \cdot \lambda)\* v = \frac{1}{\lambda} \cdot (\lambda \cdot v) = \mathcal{O}$ \Lightning
	\item[e)]
	\begin{align*}
	v,w \text{ linear abhängig } &\Leftrightarrow v = \lambda w \text{ , für ein } \lambda \in \R \\
	&\Leftrightarrow v \in \vecspaceR{w}
	\end{align*}
	\item[f)]In $V = \mathcal{F}(\R{, \R}) = \{ f: \R \rightarrow \R \vert \text{ f Abbildung} \} $ sind die Vektoren
	\begin{itemize}
		\item $f(x) = x,\quad g(x) = x^2 $ linear unabhängig
		\item $f(x) = \sin^2(x),\quad g(x) = \cos^2(x), \quad h(x) = 2$ linear abhängig: \\
		\begin{align*}
		2&=2\*(\sin^2x+\cos^2x)\\
		&=2\sin^2x+2\cos^2x\\
		0&=\underbrace{2}_{\lambda_1}\sin^2x+\underbrace{2}_{\lambda_2}\cos^2x\underbrace{-1}_{\lambda_3}\*2
		\end{align*}
	\end{itemize}
\end{itemize}
	\subsection{Satz}
$M = \{v_1,...,v_n \} \subseteq V$\\
\begin{itemize}
	\item[(i)] $M$ linear unabhängig $\Leftrightarrow  \text{ Zu jedem } v \in \vecspaceR{M}$ gibt es eindeutig bestimmte \\
	\noindent\hspace*{43mm}$\lambda_1, ... \lambda_n  \in \R:v = \sum_{i= 1}^{n} \lambda_i \cdot v_i$
	\item[(ii)] $M$ linear unabhängig, $v \notin \vecspaceR{M} \Rightarrow M \cup \{ v \} $ linear unabhängig 
\end{itemize}
\subsubsection*{Beweis}
\begin{itemize}
	\item[(i)] 
	\begin{itemize}
		\item[($\Leftarrow$)] $ \mathcal{O} \in \vecspaceR{M} \Rightarrow \exists$ eindeutig bestimmte $\lambda_1, ... , \lambda_m \in \R:$\\
		\noindent\hspace*{23mm}$\mathcal{O} = \lambda_1 \cdot v_1 + ... + \lambda_n \cdot v_n$ \\
		Gleichung erfüllt für $\lambda_1 = ... = \lambda_n = 0$ (eindeutige Lösung)
		\item[($\Rightarrow$)] Sei $M$ linear unabhängig, $v \in \vecspaceR{M}$\\
		Angenommen $v = \sum_{i = 1}^{n} \lambda_i \cdot v_i = \sum_{i = 1}^{n} \mu_i \cdot v_i $\\
		\noindent\hspace*{25mm}$\Leftrightarrow  \sum_{i=1}^{n} \underbrace{(\lambda_i - \mu_i)}_{=0\textrm{, da }M\textrm{ linear unabhängig}} \cdot v_i = \mathcal{O}$\\
		\noindent\hspace*{25mm}$\Rightarrow \lambda_i = \mu_i \qquad \forall i = 1,...,n$
	\end{itemize}
	\item[(ii)] Z.z.: $\sum_{i=1}^{n} \lambda_i \cdot v_i + \lambda \cdot v = \mathcal{O} \Rightarrow \lambda_i = 0 \quad\forall i, \lambda = 0$ \\
	Annahme: $\lambda \neq 0 \Rightarrow v = \underbrace{-\frac{\lambda_1}{\lambda}}_{\in\R} \cdot v_1 - ... - \frac{\lambda_n}{\lambda} \cdot v_n $\\
	\noindent\hspace*{29mm}$\Rightarrow v \in \vecspaceR{M}$\Lightning. Also $\lambda = 0$ \\
	$\lambda_i = 0$, weil M linear unabhängig.\hfill$\square$
\end{itemize}
\subsection{Satz}
$M \subseteq V$ linear unabhängig genau dann, wenn gilt:
\begin{align*}
	N \subseteq M,\quad \vecspaceR{N} &= \vecspaceR{M} \Rightarrow N = M 
\end{align*}
In Worten: Man kann von $M$ keinen Vektor weglassen, ohne dass der von $M$ aufgespannte Raum sich verkleinert.  \\
\subsubsection*{Beweis}
\begin{itemize}
	\item[$(\Rightarrow)$] Sei $M \subseteq V$ linear unabhängig. \\
	Angenommen: Man kann doch aus $M$ Vektoren weglassen, d.h.
	\begin{align*}
		N\subseteq M,\quad\vecspaceR{N}=\vecspaceR{M}\textrm{ und }N\neq M
	\end{align*}
	\begin{align*}
		N\neq M&\Rightarrow\exists x\in M\setminus N&\textrm{(da }N\subseteq M\textrm{)}\\
		&\Rightarrow\exists v_1,...,v_n\in N&\textrm{paarweise verschieden und}\\
		&\qquad\exists\lambda_1,...,\lambda_n\in\R&\textrm{so dass}\\
		&\qquad x=\lambda_1v_1+...+\lambda_nv_n&\textrm{(da }\vecspaceR{N}=\vecspaceR{M}\textrm{)}\\
		&\Rightarrow\lambda_1v_1+...+\lambda_nv_n-x=\mathcal{O}\\
		&\qquad\underbrace{v_1,...,v_n}_{\in N},\quad \underbrace{x}_{\in M\setminus N}\textrm{ paarweise verschieden}
	\end{align*}
	Da $N\subseteq M$, ist $\underbrace{v_1,...,v_n,x}_{\textrm{linear abhängig}}\in M\Rightarrow M$ linear abhängig\Lightning\\
	Also muss $N=M$ gelten.
	\item[$(\Leftarrow)$] Sei $M$ linear abhängig. \\
	Z.z. Man kann Vektoren aus $M$ weglassen, d.h.:
	\begin{align*}
		\exists N \subseteq M,\quad \vecspaceR{N} = \vecspaceR{M}\textrm{ und }N \neq M
	\end{align*}
	$M$ linear abhängig $\Rightarrow \exists n \in \N\quad \exists v_1, ..., v_n \in M$\\
	\noindent\hspace*{37mm}$\exists \lambda_1,...,\lambda_n \in \R $ (mit $\lambda_i \neq 0$ für ein i) \\
	\noindent\hspace*{37mm}$\lambda_1 \cdot v_1 + ... + \lambda_n \cdot v_n = 0$\\
	\\
	O.B.d.A: $\lambda_1 \neq 0 \Rightarrow v_1 = -\frac{\lambda_2}{\lambda_1} \cdot v_2 - \frac{\lambda_3}{\lambda_1} \cdot v_3 - ... -\frac{\lambda_n}{\lambda_1} \cdot v_n$\\
	Setze $N = M \setminus \{v_1 \} \Rightarrow N \neq M$\\
	Da $v_1$ Linearkombination von $v_2,...,v_n$ folgt:\\
	 Jede Linearkombination von $v_1,...,v_n$ lässt sich ausdrücken als Linearkombination von $v_2,...,v_n \Rightarrow \vecspaceR{N} = \vecspaceR{M}$\hfill$\square$
\end{itemize}
\subsection*{Basis und Dimension}
\marginpar{25.10.16}
Ein minimales Erzeugendensystem heißt Basis.
\subsection{Definition (Basis)}
$V$ \underline{endlich} erzeugter $\R$-VR. Eine endliche Menge $B\subseteq V$ heißt \underline{Basis}, falls
\begin{itemize}
	\item $\langle B\rangle_\R=V$ und
	\item $B$ linear unabhängig.
\end{itemize}
Für $V=\{\mathcal{O}\}$ ist $B=\emptyset$ die Basis.
\subsection{Beispiel}
\begin{itemize}
	\item[a)] $\{e_1,...,e_n\}$ ist Basis von $\R^n$ ('Standard-/kanonische Basis')
	\item[b)] Basisi ist nicht eindeutig.\\
	$B_1=\{\vec{2}{1}{0},\vec{2}{0}{1}\},\qquad B_2=\{\vec{2}{1}{1},\vec{2}{1}{2}$\\
	$\Rightarrow \langle B_1\rangle_\R=\langle B_2\rangle_\R$, da: $\vec{2}{1}{0}=2\vec{2}{1}{1}-\vec{2}{1}{2}$ und $\vec{2}{0}{1}=\vec{2}{1}{2}-\vec{2}{1}{1}$\\
	$\Rightarrow\vec{2}{1}{0},\vec{2}{0}{1}\in\langle B_2\rangle_\R\Rightarrow\R^2=\langle B_1\rangle_\R\subseteq\langle B_2\rangle_\R$
\end{itemize}
\subsection{Satz (Existenz von Basen)}
$V$ \underline{andlich} erzeugter $\R$-VR $\Rightarrow$ Jedes endliche Erzeugendensystem enthält Basis.
\subsubsection*{Beweis}
Sei $M\subseteq V$ endlich, $\langle M\rangle_\R=V$
\begin{itemize}
	\item $M$ linear unabhängig $\rightarrow$ fertig
	\item $M$ linear abhängig $\overset{1.17}{\Rightarrow}$ Man kann aus $M$ einen Vektor $v\in M$ weglassen,\\
	\noindent\hspace*{38mm} so dass $\langle M\setminus\{v\}\rangle_\R=V=\langle M\rangle_\R$. Nach endlich vielen\\
	\noindent\hspace*{38mm} Schritten liefert das Verfahren eine Basis.\hfill$\square$
\end{itemize}
\subsection*{Fragen}
\begin{itemize}
	\item Basis nicht eindeutig. Sind alle Basen gleich groß?
	\item geg. $w=\vec{3}{\frac{1}{3}}{0}{1}\in\R^3,\qquad S=\{e_1,e_2,e_3\}$. Wie kann man $w$ zu einer Basis ergänzen? Welche Vektoren aus $S$ sind geeignet?
	\begin{align*}
		w=\frac{1}{3}e_1+e_3&=\{\underbrace{w,e_1,e_3}_{\textrm{linear abhängig}}\}\textrm{ keine Basis, aber}\\
		&\quad\{\underbrace{w,e_1,e_2}_{\textrm{linear unabhängig}}\}\textrm{ Basis und }\{w,e_2,e_3\}\textrm{ Basis}
	\end{align*}
\end{itemize}
\subsection{Satz (Austauschlemma)}
$V$ endlich erzeugter $\R$-VR. Gegeben: $w\in V,\quad w\neq\mathcal{O},\quad w=\sum_{i=1}^{n}\lambda_iv_i$, wobei $B=\{v_1,...,v_n\}\subseteq V$ Basis von $V$.\\
$\Rightarrow\underbrace{(B\setminus\{v_j\})\cup\{w\}}_{(\star)}$ Basis, falls $\underbracket{\lambda_j\neq0}$
\subsubsection*{Beweis}
Z.z: ($\star$) ist Basis.
\begin{itemize}
	\item[1)] ($\star$) ist linear unabhängig.\\
	Z.z: \begin{align*}
		\sum_{i\neq j}^{}\mu_iv_i+\mu w=0&\Rightarrow\mu_i=0\textrm{ und }\mu=0\\
		\\
		\sum_{i\neq j}^{}\mu_iv_i+\mu w&=\sum_{i\neq j}^{}\mu_iv_i+\mu(\sum_{i=1}^{n}\lambda_iv_i)\\
		&=\sum_{i\neq j}^{}(\mu_i+\mu\lambda_i)v_i+\mu\lambda_jv_j\\
		&=0\\
		\\
		B=\{v_1,...,v_n\}\textrm{ Basis }&\Rightarrow\mu\lambda_j=0\textrm{ und }\mu_i+\mu\lambda_i=0\quad\forall i\neq j\\
		\underbracket{\lambda_j\neq 0}&\Rightarrow\mu=0\Rightarrow\mu_i+\underbrace{\mu\lambda_i}_{=0}=\mu_i=0\quad\forall i\neq j
	\end{align*}
	\item[2)] ($\star$) erzeugt $V$.\\
	\begin{align*}
		&&w&=\lambda_jv_j+\sum_{i\neq j}^{\lambda_iv_i}&&|\colon\lambda_j\textrm{, da }\lambda_j\neq 0\\
		&\Leftrightarrow &v_j&=\frac{1}{\lambda_j}w-\sum_{i\neq j}^{}\frac{\lambda_i}{\lambda_j}v_i&\\
		&\Rightarrow &v_j&\in\langle(B\setminus\{v_j\})\cup\{w\}\rangle_\R&\\
		&\Rightarrow&\langle(B\setminus\{v_j\})\cup\{w\}\rangle_\R&=\langle B\cup\{w\}\rangle_\R=V&
	\end{align*}
\end{itemize}
\subsection{Satz (Steinitz'scher Austauschsatz)}
Geg. $w_1,...,w_m\in V$ linear unabhängig, $\{v_1,...,v_n\}$ Basis von $V$.\\
Es folgt:
\begin{itemize}
	\item[a)] Aus den $n$ Vektoren $v_1,...,v_n$ kann man $n-m$ Vektoren auswählen, die mit $w_1,...,w_m$ eine Basis bilden.
	\item[b)] $m\leq n$
\end{itemize}
\subsubsection*{Beweis}
\begin{itemize}
	\item[a)] \begin{itemize}
		\item[1)] $w_1\in V\Rightarrow w_1=\sum_{i=1}^{n}\lambda_iv_i$\\
		Wären alle $\lambda_i=0$, dann wäre auch $w_1=0$. Da $\mathcal{O}\in V$ linear abhängig ist, wäre also auch $w_1,...,w_m$ linear abhängig. $\Lightning$\\
		Also: Mindestens ein $\lambda_i\neq 0$\\
		O.B.d.A. $\lambda_1\neq 0$ (sonst umnummerieren) $\overset{\textrm{1.20}}{\Rightarrow}\{w_1,v_2,...,v_n\}$ ist Basis von $V$
		\item[2)]  $w_2\in V\Rightarrow \mu_1w_1+\sum_{i=2}^{n}\mu_iv_i$\\
		Wären alle $\mu_2,...,\mu_n=0$, so wäre $w_2=\mu_1w_1$, also auch $w_1,w_2$ linear abhängig. $\Lightning$, da $\{w_1,...,w_m\}$ linear unabhängig.\\
		$\Rightarrow$ Mindestens ein $\mu_i\neq 0,\quad i\in\{2,...,n\}$\\
		O.B.d.A. $\mu_2\neq 0\overset{\textrm{1.20}}{\Rightarrow}\{w_1,w_2,v_3,...,v_n\}$ Basis von $V$
	\end{itemize}\hfill$\square$
	\item[b)] $\rightarrow$ Übung
\end{itemize}
\subsection{Korollar}
$V$ endlich erzeugter $\R$-VR
\begin{itemize}
	\item[i)] Je zwei Basen von $V$ enthalten gleich viele Elemente.
	\item[ii)] Basisergänzungssatz\\
	Jede linear unabhängige Teilmenge von $V$ lässt sich zu einer Basis von $V$ ergänzen.
\end{itemize}
\subsubsection*{Beweis}
\begin{itemize}
	\item[i)] $B,\tilde{B}$ Basen\\
	$B$ linear unabhängig $\overset{\textrm{1.22b)}}{\Rightarrow}|B|\leq|\tilde{B}|$\\
	$\tilde{B}$ linear unabhängig $\overset{\textrm{1.22b)}}{\Rightarrow}|\tilde{B}|\leq|B|$\\
	$\Rightarrow|B|=|\tilde{B}|$
	\item[ii)] Wähle beliebige Basis von $V$ und tausche aus(1.22a)).
\end{itemize}
\subsection{Satz}
$V$ endlich erzeugter $\R$-VR, $B\subseteq V$.\\
Dann sind äquivalent:
\begin{itemize}
	\item[i)] $B$ ist Basis
	\item[ii)] $B$ ist maximale linear unabhängige Menge in $V$
	\item[iii)] $B$ ist minimales Erzeugendensystem
\end{itemize}
\subsubsection*{Beweis}
\begin{itemize}
	\item[i)$\Rightarrow$ii)] Wegen 1.23 (linear unabhängige Menge zu Basis ergänzen, alle Basen gleich groß)
	\item[ii)$\Rightarrow$i)] (Bzw. $\neg$i)$\Rightarrow\neg$ii).) $B$ keine Basis, $B$ linear unabhängig\\
	 $\Rightarrow\langle B\rangle_\R\subsetneq V\Rightarrow\exists v\in V\setminus\langle B\rangle_\R\colon B\cup\{v\}$ linear unabhängig
	\item[i)$\Rightarrow$iii)] Satz 1.17 
\end{itemize}\hfill$\square$
\subsection{Definition (Dimension)}
\marginpar{26.10.16}
$V: \R $-VR\\
\begin{itemize}
	\item[i)] Ist $V$ endlich erzeugbar, $B$ Basis von $V,\quad|B| = n$ so hat $V$ die Dimension $n$, $\dim(V) = n$
	\item[ii)] Ist $V$ nicht endlich erzeugbar, so heißt $V$ unendlichdimensional. 
\end{itemize}
\subsection{Korollar}
dim $V = n, B \subseteq V, |B| = n$.\\
Dann ist $B$ Basis von $V$, wenn $B$ linear unabhängig oder $\vecspaceR{B} = V$
\subsubsection*{Beweis} Folgt aus 1.24
		\subsection{Beispiel}
\begin{itemize}
	\item[a)] $\{e_1,...,e_n\}$ Basis von $\R^n \Rightarrow \dim( \R^n) = n$
	\item[b)] $\vecspaceR{\emptyset} = \{\mathcal{O}\} \Rightarrow dim(\{\mathcal{O}\}) = 0$
	\item[c)] Bilden $\vec3{1}{0}{0}, \vec3{0}{1}{0}, \vec3{0}{1}{1}$ Basis von $V$? \\
	Ja, weil linear unabhängig (siehe Korollar 1.26).
	\item[d)] $V = \R^4, U = \vecspaceR{u_1 = \vec4{1}{2}{0}{1},u_2 = \vec4{0}{2}{1}{0}}$ \\
	$u_1, u_2$ linear unabhängig $\Rightarrow \dim(U) = 2$ \\
	Ergänze $u_1,u_2$ zu Basis von $V = \R^4$ \\
	\begin{itemize}
		\item 1. Möglichkeit (Austauschlemma + Steinitz) \\
		$\{e_1,e_2,e_3,e_4 \}$ Basis von $\R^4$ \\
		$u_1 = \vec4{1}{2}{0}{1} = e_1 + 2e_2 + e_4 \Rightarrow \{u_1, e_2,e_3, e_4\}$ Basis von $\R^4$ \\
		$u_2 = \vec4{0}{2}{1}{0} = 2 e_2 + e_3 \Rightarrow \{u_1,u_2, e_3,e_4 \}$ Basis von $\R^4$ \\
		(Basis könnte auch anders aussehen, nur beispielhaft dargestellt)
		\item 2. Möglichkeit (1.16) \\
		\begin{itemize}
			\item $e_1 \notin U$ ($\star$)(nachrechnen) \\
			$\overset{1.16}{\Rightarrow} \{u_1,u_2,e_1\}$ linear unabhängig
			\item $e_4 \notin \vecspaceR{\{u_1,u_2,e_1\}}$ (nachrechnen) \\
			$\overset{1.16}{\Rightarrow} \{u_1,u_2,e_1,e_4\}$ linear unabhängig und damit Basis (Korollar 1.26)
		\end{itemize}
		($\star$) Angenommen: 
		\begin{align*}
		e_1 = \vec4{1}{0}{0}{0} &= \lambda_1 \cdot u_1 + \lambda_2 \cdot u_2 \\
		\vec4{1}{0}{0}{0} &= \lambda_1 \vec4{1}{2}{0}{1} + \lambda_2 \vec4{0}{2}{1}{0} \\
		&\Leftrightarrow 
		\begin{cases}
		I\qquad &1 = \lambda_1 \\
		II\qquad&0 = 2 \lambda_1 + 2 \lambda_2 \\
		III\qquad&0 = \lambda_2  \\
		IV\qquad&0 = \lambda_1 \textrm{\qquad \Lightning\quad zu I} 								
		\end{cases} 
		\\
		&\Rightarrow e_1 \notin \vecspaceR{\{u_1,u_2\}} \Rightarrow \{u_1,u_2,e_1\} \text{ linear unabhängig}
		\end{align*}
	\end{itemize}
\end{itemize}
		\subsection{Satz (Dimensionssatz)}
$V\quad\R$-VR, $\dim(V) = n$ \\
\begin{itemize}
	\item[i)] $U \subseteq V $ ist UVR $ \Rightarrow \dim(U) \leq n$
	\item[ii)] $U \subseteq W \subseteq V,\qquad U,W \textrm{ sind } UVR$ mit $\dim(U) = \dim(W) \Rightarrow U = W$
	\item[iii)] $\dim(U+W) = \dim(U) + \dim(W) - \dim(U \cap W)$
\end{itemize}	
\subsubsection*{Beweis}
\begin{itemize}
	\item[i)] Basis von $U$ kann man zu Basis von $V$ ergänzen $\Rightarrow \dim(U) \leq \dim(V)$
	\item[ii)] $\dim(U)= \dim(W) \overset{U\subseteq W}{\Rightarrow}$ Basis von $U$ auch Basis von $W \Rightarrow U = W$
	\item[iii)] Sei $\{v_1,...,v_k\}$ Basis von $U \cap W$ \\
	Ergänze $\{v_1,...,v_k\}$ zu 
	\begin{itemize}
		\item[a)] Basis $\{v_1,...,v_k, u_{k+1},...,u_m\}$ von $U$
		\item[b)] Basis $\{v_1,...,v_k,w_{k+1},...,w_l \}$ Basis von $W$
	\end{itemize}
	\uline{Behauptung:} $B = \{v_1,...,v_k,w_{k+1},...,w_l,u_{k+1},...,u_m\}$ Basis von $U+W$ \\
	\begin{itemize}
		\item[1)] $B$ linear unabhängig\\
		Sei\\
		$\overbrace{\lambda_1 v_1 + ...+ \lambda_k v_k}^{=v} + \overbrace{\mu_{k+1} u_{k+1} + ... + \mu_m u_m}^{=u} + \overbrace{\gamma_{k+1} w_{k+1} +... + \gamma_l w_l}^{=w} = 0$\\
		$ \qquad \lambda_i, \mu_j, \gamma_r \in \R$ \\
		\\
		Es ist $w \in U \cap W$, da
		\begin{itemize}
			\item $w = \underbrace{\gamma_{k+1} w_{k+1}}_{\in W} + ... + \underbrace{\gamma_l w_l}_{\in W} \in W $
			\item $w = -\underbrace{u}_{\in U}- \underbrace{v}_{\in U} \in U$ \\
			Also: $w \in U \cap W$.
			\\
			\\
			$\Rightarrow \exists \alpha_1,...,\alpha_k \in \R: w = \alpha_1 v_1 + ... + \alpha_k v_k$ \\
			$\Rightarrow w = \gamma_{k+1} w_{k+1} + ... + \gamma_l w_l = \alpha_1 v_1 + ... + \alpha_k v_k$ \\
			$\Rightarrow  \gamma_{k+1} w_{k+1} + ... + \gamma_l w_l - \alpha_1 v_1 - ... - \alpha_k v_k = 0$ \\
			$\{v_1,...,v_k,w_{k+1} , ... , w_l\}$ linear unabhängig\\
			$\Rightarrow 
			\gamma_{k+1} = ... = \gamma_l = \alpha_1 = ... = \alpha_k = 0$\\
			$\Rightarrow w = \mathcal{O}$ und $ v + u + w = v + u = \lambda_1 v_1 + ... + \lambda_k v_k + \mu_{k+1}u_{k+1} + ... + \mu_m u_m = 0$\\
			$\{v_1, ..., v_k, u_{k+1}, ... , u_m \}$ linear unabhängig (Basis von $U$) \\
			$\Rightarrow \lambda_1 = ... = \lambda_k = \mu_{k+1} = ...= \mu_m = 0$
		\end{itemize}
		\item[2)] $\vecspaceR{B} = U + W$, da: 
		\begin{itemize}
			\item $\vecspaceR{B} \subseteq U + W$ (da $\underbrace{u+v}_{\in U} + \underbrace{w}_{\in W} \in U + W$) 
			\item $U \subseteq \vecspaceR{B}$ (da Basis von $U$ in $B$) \\
			\item$W \subseteq \vecspaceR{B} $
		\end{itemize}
		$\Rightarrow U + W  \subseteq \vecspaceR{B}$
		
	\end{itemize}
\end{itemize}\hfill$\square$
\subsection{Bemerkung (Koordinaten)}
Geg.: Basis $\{v_1,...,v_n\}$ von $V$,	 Vektor $u \in V$ \\
$\Rightarrow u = \lambda_1 v_1 + ... + \lambda_n v_n$\\
$\lambda_i$ eindeutig und heißen Koordinaten von $u$ bezüglich der Basis $B$. \\
z.B.: $\vec3{2}{1}{3} = \vec3{1}{0}{0} + \vec3{0}{1}{0} + 3 \vec3{\frac{1}{3}}{0}{1} \Rightarrow \vec3{2}{1}{3}$ hat Koordinaten 1,1,3 bezüglich $B = \{\vec3{1}{0}{0}, \vec3{0}{1}{0}, \vec3{\frac{1}{3}}{0}{1}\}$

%%2. Matrizen und lineare Gleichungssysteme%%
%%2. Matrizen und lineare Gleichungssysteme%%
%%2. Matrizen und lineare Gleichungssysteme%%
\newpage
\section{Matrizen und lineare Gleichungssysteme}
\marginpar{02.11.16}
\subsection{Beispiel}
\begin{itemize}
	\item Ein Bauer besitzt Kühe und Gänse
	\item Insgesamt 18 Tiere mit 40 Beinen
	\item Frage: Wieviele der Tiere sind Kühe?
\end{itemize}
\uline{Lineares Gleichungssystem (LGS):}
$\ast
\begin{cases}
I:\qquad k + g &= 18 \\
II: \quad4k + 2g &= 40\qquad \Leftrightarrow\quad 2k +g = 20 \\
\end{cases}$ \\
\noindent\hspace*{49mm}
$\Rightarrow g = 20 -2k = 18-k \Leftrightarrow k = 2 \Rightarrow g = 16$\\
\\
\uline{Vektorenschreibweise von $\ast$:} \\
$\vec2{k+g}{4k+2g} = \vec2{18}{40}$ oder $k\vec2{1}{4} + g\vec2{1}{2} = \vec2{18}{40}$\\
\\
\uline{Matrixschreibweise:} \\
$\underbrace{\begin{pmatrix}
1&1 \\
4&2 \\
\end{pmatrix}}_{\textrm{Matrix}} \cdot \vec2{k}{g} = \vec2{18}{40}$
\subsection{Definition (Matrix)}
Allgemeines lineares Gleichungssystem: \\
Gegeben:
\begin{itemize}
	\item Unbekannte $x_1,...,x_n \in \R, n \in \mathbb{N}$
	\item $m \in \mathbb{N}$ Gleichungen
	\item Koeffizienten $a_{ij} \in \R, i = 1,...,m; j = 1,...,n$
\end{itemize}
$a_{11}x_1 + a_{12}x_2 + ... + a_{1n}x_n = b_1$\\
$a_{21}x_1 + a_{22}x_2 + ... + a_{2n}x_n = b_2$\\
$\vdots \qquad \qquad \vdots \quad \qquad \vdots \qquad \quad \vdots\quad\qquad\vdots$\\
$a_{m1}x_1 + a_{m2}x_2 + ... + a_{mn}x_n = b_m$\\
\\
\\
\\
\\\\
\uline{Matrixschreibweise:} \\
$A x = b$ mit \\
\begin{itemize}
	\item$A=\underset{\textrm{Spalte}}{\begin{pmatrix}
	a_{11}&a_{12}&\cdots&a_{1n}\\
	a_{21}&a_{22}&\cdots&a_{2n}\\
	\vdots&\vdots&\ddots&\vdots\\
	a_{m1}&\underset{\uparrow}{a_{m2}}&\cdots&a_{mn}
	\end{pmatrix}}\leftarrow\textrm{Zeile}$
	\item
	$x = \vec3{x_1}{\vdots}{x_n} \in \R^n$
	\item$b = \vec3{b_1}{\vdots}{b_m} \in \R^m$
\end{itemize}
	Man schreibt $A = (a_{ij})_{\substack{i=1,...,m\\j = 1,...,n}}$ oder nur $A =(a_{ij})$, wenn $m,n$ schon bekannt.
	\begin{itemize}
	\item $a_{ij} \in \R$ - \uline{Eingänge} der Matrix $A$
	\item $A$ - reelle $m\times n$- Matrix
	\item $\mathcal{M}_{m,n}(\R)$ - Menge aller reellen $m \times n$ - Matrizen
	\item $\mathcal{M}_{n,n}(\R) = M_n (\R)$ - quadratische Matrizen 
\end{itemize}
$(\ast\ast)\quad$Dabei ist\\
$Ax\coloneqq x_1 \vec3{a_{11}}{\vdots}{a_{m1}} + x_2 \vec3{a_{12}}{\vdots}{a_{m2}} + \cdots + x_n \vec3{a_{1m}}{\vdots}{a_{mn}} =
\begin{pmatrix}
a_{11}x_1 + a_{12}x_2 + ... + a_{1n}x_n \\
\vdots\quad+\quad\vdots\quad+\quad\vdots\quad+\quad\vdots \\
a_{m1}x_1 + a_{m2}x_2 + ... + a_{mn}x_n \\
\end{pmatrix} \in \R^m$
\subsection{Bemerkung}
Aus $(\ast\ast)$ ergibt sich: $A: \R^n \rightarrow \R^m , x \longmapsto A \cdot x$ für $A \in \mathcal{M}_{m,n}(\R)$\\
$A$ bildet Vektoren auf Vektoren ab.\\
\\\\\\\\
Matrizen können nicht nur zur Lösung von LGS verwendet werden, sondern auch in der Geometrie: \\
\subsection{Beispiel:}
\begin{itemize}
	\item[a)] Spiegelung $S_y$ ain $\R^2$ an $y$-Achse\\
	\begin{minipage}[c]{0.5\textwidth}
		\InitGraph{6}{3.5}{2.5}{0}{1cm}
		\Coordinates(4,2)(4,2)
		\SetDarkgrey
		\TextAt(3.5,0)[b]{$x$}
		\TextAt(0,3.5)[l]{$y$}
		\MoveTo(0,0)
		\PaintTriangle(2,0)(0,2)(0,0)
		\Text[tr]{$D$}
		\SetLightgrey
		\PaintTriangle(-2,0)(0,2)(0,0)
		\Text[tl]{$D'$}
		\SetBlack
		\MoveTo(1.5,1.5)
		\Text[r]{$\vec2{x}{y}$}
		\MoveTo(1,2)
		\Text[t]{$S_y$}
		\Bezier(-1.5,1.5)(1,3)(1.5,1.5)
		\MoveTo(-1.5,1.5)
		\ArrowDirection(200,0.1)
		\Text[l]{$\vec2{-x}{y}$}
		\CloseGraph
	\end{minipage}
	\begin{minipage}[c]{0.5\textwidth}
		
		$S_y: \vec2{x}{y} \mapsto \vec2{-x}{y}\quad x,y \in \R$\\
		$S_y: \begin{pmatrix}
		s_{11} &s_{12} \\
		s_{21} &s_{22} \\
		\end{pmatrix} \\
		S_y \vec2{x}{y} = \begin{pmatrix}
		s_{11} +s_{12} \\
		s_{21} +s_{22} \\
		\end{pmatrix} = \vec2{-x}{y}$\\
		$\Rightarrow s_{11} = -1\quad s_{12} = 0\quad s_{21} = 0\quad s_{22} = 1\\
		 S_y = \begin{pmatrix}
		-1 &0 \\
		0 &1 \\
		\end{pmatrix}$\\
		$S_y$ bildet $D$ auf $D'$ ab.
	\end{minipage}
	\item[b)] Drehung $D_\phi$ um $\phi \in [0,2\pi)$ \\
	Vorüberlegung am Einheitskreis:\\
	\\
	
	\InitGraph{15}{4.5}{0}{0}{1cm}
	\OpenWindowAt(1,1)(5,4)(1,1)
	\TextAt(3,1.5)[c]{$\vec2{1}{0} \overset{D_{\phi}}{\rightarrow}\vec2{\cos\phi}{\sin\phi}$}
	\Axes
	\SetDotted
	\CircleAt(0,0)(2)
	\SetNormal
	
	\ArrowAt(0,0,2,0)
	\ArrowAt(0,0,1,1.72)
	\EllipticArcAt(0,0)(1,1)(0,60)
	\TextAt(0.2,0)[tr]{$\phi$}
	\SetDashed
	\LineAt(0,-1,1,-1)
	\TextAt(0.5,-0.6)[c]{$\cos\phi$}
	\LineAt(-1,0,-1,1.72)
	\TextAt(-1,1)[c]{$\sin\phi$}
	\SetNormal
	\CloseWindow
	\OpenWindowAt(7,1)(5,4)(4,1)
	\TextAt(-3,2.5)[c]{$\vec2{0}{1}\overset{D_{\phi}}{\rightarrow}\vec2{-\sin\phi}{\cos\phi}$}
	\Axes
	\SetDotted
	\CircleAt(0,0)(2)
	\SetNormal
	
	\ArrowAt(0,0,0,2)
	\ArrowAt(0,0,-1.72,1)
	\EllipticArcAt(0,0)(1,1)(90,60)
	\TextAt(0,0.2)[tl]{$\phi$}
	\SetDashed
	\LineAt(0,-1,-1.72,-1)
	\TextAt(-0.8,-0.6)[c]{$-\sin\phi$}
	\LineAt(1,0,1,1.72)
	\TextAt(1,1)[c]{$\cos\phi$}
	\SetNormal
	
	\CloseWindow
	\CloseGraph
\\
	\begin{minipage}[c]{0.5\textwidth}
	 \InitGraph{6}{3.5}{2.5}{0}{1cm}
	 \Coordinates(4,2)(4,2)
	 \TextAt(3.5,0)[b]{$x$}
	 \TextAt(0,3.5)[l]{$y$}
	 \MoveTo(0,0)
	 \Text[tr]{$\phi$}
	 \MoveTo(1.5,1.5)
	 \Text[r]{$\vec2{x}{y}$}
	 \MoveTo(1,2)
	 \Text[t]{$D_\phi$}
	 \Bezier(-1.5,1.8)(1,3)(1.5,1.5)
	 \MoveTo(-1.5,1.8)
	 \ArrowDirection(200,0.1)
	 \Text[l]{$\vec2{x'}{y'}$}
	 \ArrowAt(0,0,2,1)
	 \ArrowAt(0,0,-1.5,1.5)
	 \EllipticArcAt(0,0)(0.8,0.8)(30,105)
	 \CloseGraph
	\end{minipage}
	\begin{minipage}[c]{0.5\textwidth}
			$D_{\phi}: \vec2{x}{y} \rightarrow \vec2{x'}{y'}$\\
		$D_{\phi}= \begin{pmatrix}
		d_{11} &d_{12} \\
		d_{21} &d_{22} \\
		\end{pmatrix}\\
		\quad\Rightarrow D_{\phi}  \vec2{1}{0} = \vec2{d_{11}}{d_{21}} = \vec2{\cos \phi}{\sin\phi}$\textrm{ und }\\
		$D_\phi \vec2{0}{1} = \vec2{d_{12}}{d_{22}}= \vec2{-\sin\phi}{\cos\phi}$\\
		$\Rightarrow D_\phi = (D_\phi \cdot e_1, D_\phi \cdot e_2) = \begin{pmatrix}
		\cos\phi &- \sin\phi \\
		\sin\phi &\cos\phi \\
		\end{pmatrix}$
	\end{minipage}
\end{itemize}
\subsection{Bemerkung}
Aus Beispiel 2.4 b) und Def 2.2 ergibt sich: \\
$A \cdot e_j = 1 \cdot \vec3{a_{1j}}{\vdots}{a_{mj}}\quad$ ($j$-te Spalte von $A\in\mathcal{M}_{m,n}(\R)$) \\
$\Rightarrow A = (\underbrace{A_{e_1}, A_{e_2},...,A_{e_n}}_{\textrm{Spalten}})$
\subsection{Satz}
$A \in \mathcal{M}_{m,n}(\R)\qquad x,y \in \R^n$\\
\begin{itemize}
	\item[i)] $A(\lambda x) = \lambda (A \cdot x) \qquad \lambda \in \R$
	\item[ii)] $A(x+y) = Ax +  Ay$
\end{itemize}
\subsubsection*{Beweis}
\begin{itemize}
	\item[i)] \begin{align*}
	A(\lambda x) &= (\lambda x_1) \underbrace{A \cdot e_1}_{\textrm{1. Spalte}} + (\lambda x_2)A e_2 +...+ (\lambda x_n)\underbrace{A e_n}_{n\textrm{-te Spalte}}\\
	&= \lambda[x_1 (Ae_1) + ... + x_n (Ae_n)] \\
	&= \lambda (Ax)
	\end{align*}
	\item[ii)] Übung
\end{itemize}
\subsection{Beispiel (Folien 02.11.2016)}
\begin{itemize}
	\item[a)] $A \cdot x = (D_\pi \circ S_y) \cdot x = D_\pi \vec2{-x_1}{x_2} = \begin{pmatrix}
	-1& 0 \\
	0 &1 \\
	\end{pmatrix} \vec2{-x_1}{x_2} = \vec2{x_1}{-x_2}\\
	\Rightarrow \vec2{x_1}{x_2} \overset{A}{\mapsto} \vec2{x_1}{-x_2} \qquad A = \begin{pmatrix}
	1  &0 \\
	0 &-1 \\
	\end{pmatrix}$
	\item[b)]
	Berechnung Matrixprodukt (Verknüpfung) $A\*B$ \\
	\begin{align*}
		\underbrace{\begin{pmatrix}
			a&b\\c&d
			\end{pmatrix}}_{A}\underbrace{\begin{pmatrix}
			e&f\\g&h
			\end{pmatrix}}_{B}\vec2{x_1}{x_2}&=\begin{pmatrix}
		a&b\\c&d
	\end{pmatrix}[\underbrace{x_1\vec2{e}{g}+x_2\vec2{f}{h}}_{\in\R^2}]\\
	&\overset{2.6}{=}\quad x_1\lbrack\underbrace{e\vec2{a}{c}+g\vec2{b}{d}}_{\in\R^2}]+x_2[\underbrace{f\vec2{a}{c}+h\vec2{b}{d}}_{\in\R^2}]\\
	&=\underbrace{\begin{pmatrix}
		ea+gb&fa+hb\\ec+gd&fc+hd
		\end{pmatrix}}_{\textrm{Matrixprodukt }A\*B}\vec2{x_1}{x_2}
 	\end{align*}
\end{itemize}	
\subsection{Definition (Matrixprodukt)}
$A=(a_{ij})\in\mathcal{M}_{m,n}(\R)\qquad B=(b_{ij})\in\mathcal{M}_{m,n}(\R)$
\begin{align*}
	A\*B&=(c_{ik})\quad\in\mathcal{M}_{m,l}(\R)\\
	c_{ik}&=(i\textrm{-te Zeile von }A)\*(k\textrm{-te Spalte von }B)\\
	&=a_{i1}b_{1k}+a_{i2}b_{2k}+\cdots+a_{in}b_{nk}\\
	&=\sum_{j=1}^{n}a_{ij}b_{jk}
\end{align*}	
(Skalarprodukt)
\subsection{Beispiel}
\marginpar{08.11.16}
$A = \begin{pmatrix}
\uline{1} &\uline{0} & \uline{-1}\\
2 & -3 & 1\\
\end{pmatrix},\quad B = \begin{pmatrix}
1 & \uline{2} & -1 \\
0 & \uline{0} & 0\\
0 & \uline{1} & 0 \\
\end{pmatrix}, \qquad
A \cdot B = \begin{pmatrix}
1 & \uline{1} & -1 \\
2 & 3 & -2 \\
\end{pmatrix}$\\
$B\cdot A$ nicht definiert!
\subsection{Satz + Definition}
$\mathcal{M}_{m,n}(\R)$ ist Vektorraum mit 
\begin{itemize}
	\item $A + B = (a_{ij} + b_{ij}) \qquad A,B \in \mathcal{M}_{m,n}(\R)$
	\item $\lambda \cdot A = (\lambda a_{ij})\qquad A \in \mathcal{M}_{m,n}(\R), \lambda \in \R$
\end{itemize}
Beweis: Siehe Hausaufgabe 03 Aufgabe 4a)
\subsection{Beispiel}
$A = \begin{pmatrix}
1 & 2 & 3 \\
-1 & 0 & 2 \\
\end{pmatrix}\qquad B = \begin{pmatrix}
0& 0& -3 \\
1&0&1 \\
\end{pmatrix} \\
A + B = \begin{pmatrix}
1 & 2 & 0 \\
0 & 0 & 3 \\
\end{pmatrix}, \qquad (-2) \cdot A = \begin{pmatrix}
-2 & -4 & -6 \\
2 & 0 & -4 \\
\end{pmatrix}$
\subsection{Definition (Matrizentransponierung)}
\begin{itemize}
	\item[i)] 
	$A \in \mathcal{M}_{m,n}(\R),\quad A = (a_{ij})$.\\
	 Die zu \uline{$A$ transponierte Matrix} (Tauschen von Zeilen und Spalten):
	\begin{center}
		$A^T = \begin{pmatrix}
		a_{11} & a_{21} & \cdots & a_{m1} \\
		a_{12} & a_{22} & \cdots & a_{m2} \\
		\vdots &\vdots&\ddots&\vdots\\
		a_{1n} & a_{2n} & \cdots & a_{mn} \\
		\end{pmatrix} \in \mathcal{M}_{m,n}(\R)$
	\end{center}
	z.B.: $A = \begin{pmatrix}
	1 & 2 & 0 \\
	-1 & 1 & 2 \\
	\end{pmatrix} \Rightarrow A^T = \begin{pmatrix}
	1 & -1 \\
	2 & 1 \\
	0 & 2 \\
	\end{pmatrix} $\\
	Eine Matix heißt \uline{symmetrisch}, wenn $A = A^T$, z.B.: 
	\begin{center}
		$A=\begin{pmatrix}
		1 & 2 & 0 \\
		2 & 3 & 4 \\
		0 & 4 & -1 \\
		\end{pmatrix}$
	\end{center}
	\item[ii)]
	\begin{itemize}
		\item Nullmatrix: 
		$\mathcal{O}_{m,n} = \begin{pmatrix}
		0 & \cdots & 0 \\
		\vdots & \ddots & \vdots \\
		0 & \cdots & 0 \\
		\end{pmatrix} \in \mathcal{M}_{m,n}(\R)$
		\item Einheitsmatrix (nur Hauptdiagonale): 
		$E_n = \begin{pmatrix}
		1 & \dots & 0 \\
		\vdots& \ddots &\vdots \\
		0 & \dots& 1 \\
		\end{pmatrix} \in \mathcal{M}_{n}(\R)$
	\end{itemize}
\end{itemize}
\subsection{Beispiel}
\begin{itemize}
	\item[a)] $A = \begin{pmatrix}
	1 & 1 \\
	1 & 1 \\
	\end{pmatrix}\qquad B = \begin{pmatrix}
	2 & 0 \\
	3 & 0 \\
	\end{pmatrix}\\
	A \cdot B = \begin{pmatrix}
	5 & 0 \\
	5 & 0 \\
	\end{pmatrix} \neq B \cdot A = \begin{pmatrix}
	2 & 2 \\
	3 & 3 \\
	\end{pmatrix}$ Matrixmultiplikation nicht kommutativ!
	\item[b)]
	$A \in \mathcal{M}_{m,n}(\R)$ \\
	$A \cdot E_n = A$ und $E_m \cdot A = A$
\end{itemize}


\newpage
\section{Gruppen}
\subsection{Beispiel (Wiederholung zu Permutationen)}
Geg.: Menge $\{A,B,C\}$\\
Anordnungen: ABC, CAB, ACB, ... $\rightarrow$ $3 \cdot 2 \cdot 1= 3!$ Möglichkeiten\\
Jede Anordnung kann man auffassen als eineindeutige (bijektive) Abbildung\\
$\pi : \{A,B,C\} \rightarrow \{A,B,C\} \\
\pi: \begin{tabular}{c|c | c | c }
x&A & B & C \\ \hline
$\pi(x)$&A & C & B \\
\end{tabular}$
\subsection{Definition (Permutation)}	
\begin{itemize}
	\item Eine \uline{Permutation} ist eine eineindeutige Abbildung einer endlichen Menge auf sich selbst. Im Allgemeinen verwendet man die Menge $\{1,...,n\}$ und schreibt eine Permutation $\pi$ als Wertetabelle $\pi = \begin{pmatrix}
	1~~~...~~~n\\\pi(1)~...~\pi(n)
	\end{pmatrix}$ oder als geordnete Liste der Werte $\pi = \pi(1)... \pi(n)$ 
	\item $\mathscr{S}_n$- Menge aller Permutationen von $\{1,...,n \},\qquad | \mathscr{S}_n| = n!$
\end{itemize}
Beispiel: $\mathscr{S}_2=\{\id,(AB)\}=\{\id,(12)\},\quad|\mathscr{S}_2|=2!=2$\\ mit $\id=\begin{pmatrix}
AB\\AB
\end{pmatrix},\qquad \pi=\begin{pmatrix}
AB\\BA
\end{pmatrix}$
\subsection{Beispiel}
\begin{minipage}[c]{0.5\textwidth}
	\begin{itemize}
		\item $M = \{1,2,...,5\}$ \\
		$\pi = \pi(1)...\pi(5) = 23154$\\
		oder $\pi = (\begin{cases}
		12345 \\
		23154 \\
		\end{cases})$
		\item id(i) $= i \qquad \forall i \in \{1,..,n\}$
	\end{itemize}
\end{minipage}
\begin{minipage}[c]{0.5\textwidth}
	\begin{tikzpicture}
	
	\node (v1) at (0,0) {1};
	\node (v2) at (0.5,-1.5) {2};
	\node (v3) at (-1,-1.5) {3};
	\draw [->] (v1) edge (v2);
	\draw [->] (v2) edge (v3);
	\draw [->] (v3) edge (v1);
	\node (v4) at (2,0) {4};
	\node (v5) at (2,-1.5) {5};
	\draw [->] (v4) edge (v5);
	\draw [->](1.7454,-1.4777) .. controls (1,-1) and (1.5,-0.5) .. (1.7657,-0.1725);
	\end{tikzpicture}\\
	Graph der Permutation
\end{minipage}
\subsection{Bemerkung}
In Literatur oft \uline{Zyklenschreibweise}:\\
Zyklus $(a_1 a_2... a_k)$ bedeutet $\pi(a_i) = a_{i+1}$ und $\pi(a_k) = a_1$\\
z.B.: $\pi = (123)(45)$ 
\subsection*{Verknüpfung von Permutationen}
\subsection{Beispiel}
\begin{minipage}[c]{0.5\textwidth}
	\begin{tikzpicture}
	
	\node (v1) at (0,0) {1};
	\node (v2) at (0.5,-1.5) {2};
	\node (v3) at (-1,-1.5) {3};
	\draw [->] (v1) edge (v2);
	\draw [->] (v2) edge (v3);
	\draw [->] (v3) edge (v1);
	\node (v4) at (2,0) {4};
	\node (v5) at (2,-1.5) {5};
	\draw [->] (v4) edge (v5);
	\draw [->](1.7454,-1.4777) .. controls (1,-1) and (1.5,-0.5) .. (1.7657,-0.1725);
	\end{tikzpicture}
\end{minipage}
\begin{minipage}[c]{0.5\textwidth}
	$\pi = \begin{pmatrix}
	12345 \\
	23154 \\
	\end{pmatrix} = (123)(45)$
\end{minipage}\\
\begin{minipage}[c]{0.5\textwidth}
	\begin{tikzpicture}
	
	\node (v1) at (-2,1.5) {1};
	\node (v2) at (0,1.5) {2};
	\node (v3) at (0,-0.5) {3};
	\node (v4) at (-2,-0.5) {4};
	\node at (2,0.5) {5};
	\draw [->] (v1) edge (v2);
	\draw [->] (v2) edge (v3);
	\draw [->] (v3) edge (v4);
	\draw [->] (v4) edge (v1);
	\draw [->](1.976,0.765) .. controls (2.5,1.5) and (2.5,-0.5) .. (1.952,0.2238);
	\end{tikzpicture}
\end{minipage}
\begin{minipage}[c]{0.5\textwidth}
	$\sigma = \begin{pmatrix}
	12345 \\
	23415 \\
	\end{pmatrix} = (1234)(5)$
\end{minipage}\\
\begin{minipage}[c]{0.5\textwidth}
	\begin{tikzpicture}
	
	
	\node (v1) at (-1.5,1.5) {1};
	\node (v2) at (0.5,1.5) {3};
	\node (v3) at (2,0) {5};
	\node (v4) at (0.5,-1) {4};
	\node (v5) at (-1.5,-1) {2};
	\draw [->] (v1) edge (v2);
	\draw [->] (v2) edge (v3);
	\draw [->] (v3) edge (v4);
	\draw [->] (v4) edge (v5);
	\draw [->] (v5) edge (v1);
	\end{tikzpicture} 
\end{minipage}
\begin{minipage}[c]{0.5\textwidth}
	$\pi\sigma= \begin{pmatrix}
	12345 \\
	31524 \\
	\end{pmatrix} = (13542)$
\end{minipage}\\
\subsection{Bemerkung} 
\begin{itemize}
	\item[a)] Die Verknüpfung von 2 Permutationen $\pi, \sigma$ ist wieder Permutation $\eta$ mit $\eta(i) = \pi \circ \sigma(i) = \pi(\sigma(i))$
	\item[b)] Fixpunkte mit $\pi(i)= i$ lässt man weg, z.B. $\underbrace{(123)(4)}_{\in \mathscr{S}_4} = (123)$
	\item[c)] Jede Permutation kann als Produkt disjunkter Zyklen geschrieben werden, z.B.: $ \underset{\textrm{Verkettung }\circ}{(34)\*(345)}  = (3)(45) = (45)$.\\
	Zwei Zyklen heißen \uline{disjunkt}, wenn $\{a_1...a_k\}\cap\{b_1...b_j\}=\emptyset$.
	\item[d)] Permutationen sind nur in sehr seltenen Fällen kommutativ: \\
	$(123)(23) = (12) \neq (23)(123) = (13)$
	\item[e)] Zyklendarstellung nicht eindeutig, z.B.: \\
	$(123) = (231)$ oder $(34)(12) = (12)(34)$
\end{itemize}
\subsection{Beispiel}
\marginpar{09.11.16}
\begin{minipage}[c]{0.8\textwidth}
	\renewcommand{\arraystretch}{1.7}
	\begin{tabular}{ l | c c c c }
		$\substack{\textrm{Symmetrie-}\\\textrm{operationen des}\\\textrm{Rechtecks}}$& Identität & $\substack{\textrm{Spiegelung}\\\textrm{y-Achse}}$ & $\substack{\textrm{Spiegelung}\\\textrm{x-Achse}}$ & Drehung $180 \degree$ \\&
		\frame{\begin{tabular}{ c c }
				D & C \\ 
				A & B \\
		\end{tabular}}&
		\frame{\begin{tabular}{ c : c }
				C&D \\ 
				B&A \\
		\end{tabular}} &
		\frame{\begin{tabular}{ c c }
				
				A&B \\ \hdashline
				D&C \\ 
		\end{tabular}} &
		\frame{\begin{tabular}{ c c }
				
				B&A \\ 
				C&D \\ 
		\end{tabular}}  \\ \hline
		$\substack{\textrm{als Matrix}}$ &
		$E_2 = \begin{pmatrix}
		1 & 0 \\
		0 & 1 \\
		\end{pmatrix}$ & $S_y = \begin{pmatrix}
		-1 & 0 \\
		0 & 1\\
		\end{pmatrix}$ & $S_x = \begin{pmatrix}
		1 & 0 \\
		0 & -1 \\
		\end{pmatrix}$ & $D_\pi = \begin{pmatrix}
		-1 & 0 \\
		0 & -1 \\
		\end{pmatrix}$ \\ \hline
		$\substack{\textrm{als Permutation}\\ \textrm{der Ecken}}$ & id & 
		$\pi = (AB)(CD)$ &
		$\sigma = (AD)(BC)$ &
		$\eta = (AC)(BD)$ \\
		
	\end{tabular}
\end{minipage} \\
\\
\\
\subsubsection*{Verknüpfungstafel}
\begin{tabular}{c | c c c c}
	$\overset{Matrixmultiplikation}{\cdot} $ & $E_2$ & $S_y $ & $S_x$ & $D_\pi$ \\ \hline
	$E_2$ & $E_2$ & $S_y$ & $S_x$ & $D_ \pi$ \\
	$S_y$ & $S_y$ & $E_2$ & $D_\pi$ & $S_x$ \\
	$S_x$ & $S_x$ & $D_\pi$ & $E_2$ & $S_y$ \\
	$D_\pi$ & $D_\pi$ & $S_x$ & $S_y$ & $E_2$ \\
\end{tabular}
\subsection{Definition (Grundbegriffe)}
\begin{itemize}
	\item Seien $X,Y$ nichtleere Mengen, Eine Verknüpfung ' $\cdot$ ' ist eine Abbildung
	\begin{center}
		$X \times X \rightarrow Y\qquad (a,b) \rightarrow a \cdot b  \qquad( \leftarrow$ 'Produkt' von a und b)
	\end{center}
	\item Eine Menge $X \neq \emptyset$ heißt \uline{abgeschlossen} bzgl. einer Verknüpfung ' $\cdot$ ', falls $a \cdot b \in  X \qquad \forall a,b \in X$. \\
	Beispiel: $X=\{-1,1\}$ mit '$\*$' Addition $\Rightarrow(-1)\*(1)=-1+1=0$
\end{itemize}
	Die Menge $\{id, \pi ,\sigma, \eta \}$ aus Beispiel 3.7 ist abgeschlossen bzgl. der Verkettung von Permutationen \\
	\subsubsection*{Bemerkung} Die Verknüpfung von Elementen einer endlichen Menge stellt man anhand der Verknüpfungstafel dar, siehe Bsp. 3.7
\subsection{Definition (Gruppe)}
\begin{itemize}
	\item[a)] Eine \uline{Gruppe} ist ein Paar $(G, \cdot) $ mit Menge $G \neq \emptyset$ und einer Verknüpfung $\cdot : \underbrace{G \times G \rightarrow G}_{\text{abgeschlossen!}}$, die folgende Eigenschaften erfüllt:
	\begin{itemize}
		\item[1)] $(a \cdot b) \cdot c = a \cdot (b \cdot c) \qquad \forall a,b,c \in G\qquad~$ Assoziativität
		\item[2)] $\exists e \in G: a \cdot e = e \cdot a = a \qquad \forall a \in G\qquad$ Neutralelement
		\item[3)] $\forall a \in G\quad \exists a^{-1} \in G: a \cdot a^{-1} = a^{-1}*a = e\qquad$ Inverse
	\end{itemize}
		Falls zusätzlich 
		\begin{itemize}
			\item[4)] $a \cdot b = b \cdot a \qquad  \forall a,b \in G\qquad$ Kommutativität \end{itemize}
		gilt, dann heißt $G$ \uline{abelsche Gruppe}.
	
	\item[b)] $| G |$ heißt \uline{Ordnung} der Gruppe $G$.
\end{itemize}
\subsection{Beispiel}
\begin{itemize}
	\item[a)] $(\{e\}, \cdot )$ ist Gruppe
	\item[b)] $\R, \mathbb{Z}, \mathbb{Q}$ mit ' + ' ist abelsche Gruppe. Inverse zu a ist -a.
	\item[c)] $\R, \mathbb{Z}, \mathbb{Q}$ mit ' $\cdot$ ' keine Gruppen. Problem: 0 besitz keine Inverse, weil \\
	$0 \cdot a = 1$\Lightning
	\item[$\Rightarrow$]  $\R, \mathbb{Q}$ mit ' $\cdot$ ' Gruppen, wenn man 0 weglässt
	\item[d)] Einzige \uline{endliche} Gruppen von reellen Zahlen: 
	\begin{itemize}
		\item $(\{1\}, \cdot )$ bzw. $(\{0\}, + )$
		\item $(\{1,-1\}, \cdot)$
	\end{itemize}
	Für weitere endliche Gruppen muss man Restklassen (Beispiel 3.12) Matrizen oder Permutationen betrachten
	\item[e)] $\mathscr{S}_2 = \{\id, (12)\}$ und \\
	$\mathscr{S}_3 = \{\id, (12), (23),(13),(123),(132)\}$ sind Gruppen (s. 3.11)
	\item[f)] $V_4 = \{\id, \pi, \sigma, \eta \}$ aus Beispiel 3.7 ist die Symmetriegruppe des Rechtecks und heißt 'Kleinsche Vierergruppe' ($V_4$ Gruppe: s. 3.16 e).
\end{itemize}
\subsection{Satz}
$\mathscr{S}_n$ ist eine \uline{nicht} abelsche Gruppe. (Name: Symmetrische Gruppe) \\
\subsubsection*{Beweis}
\begin{itemize}
	\item assoziativ: $\pi, \sigma, \eta \in \mathscr{S}_n \Rightarrow \underbrace{(\pi \cdot \sigma) \cdot \eta}_{\textrm{Verknüpfung von Abbildungen}} = \overset{\textrm{bijektive Abbildungen}}{\pi \overset{\uparrow}{\cdot} (\sigma \overset{\uparrow}{\cdot} \eta)}$
	\item Neutralelement: id, denn \\
	$\id \cdot \pi = \pi \cdot \id = \pi\qquad\forall\pi\in\mathscr{S}_n$
	\item Inverse: Alle Pfeile eines Zyklus werden umgedreht, d.h. die Zyklen werden rückwärts gelesen:\\
	\begin{tikzpicture}
	
	\node (v3) at (-0.5,2) {1};
	\node (v1) at (0.5,0) {2};
	\node (v2) at (-1.5,0) {3};
	\draw [->] (v1) edge (v2);
	\draw [->] (v2) edge (v3);
	\draw [->] (v3) edge (v1);
	\node (v6) at (3.5,2) {1};
	\node (v5) at (4.5,0) {2};
	\node (v4) at (2.5,0) {3};
	\draw [->] (v4) edge (v5);
	\draw [->] (v5) edge (v6);
	\draw [->] (v6) edge (v4);
	\node (v7) at (1,1) {};
	\node (v8) at (2,1) {};
	\draw [->] (v7) edge (v8);
	\end{tikzpicture}\\
	$\pi=(123)\qquad\qquad\qquad\pi^{-1}=(132)$\\
Fixpunkte und 2er-Zyklen ändern sich dabei nicht:\\
$\sigma = (1678)(23)\Rightarrow\sigma^{-1} = (1876)(23)$\\
Setzt man die Pfeile von den Graphen $\pi$ und $\pi^{-1}$ zusammen, ändert sich nichts, d.h. $\pi \cdot  \pi^{-1}(i) = i \Rightarrow \pi \cdot  \pi^{-1} = \id = \pi \cdot  \pi^{-1}$
\item nicht abelsch: Bem. 3.6d)
\end{itemize}\newpage
\marginpar{15.11.2016}
\subsection{Beispiel}	
Restklassen modulo $n:  \mathbb{Z}_n = \{0,1,...,n-1\}$, \\
\\
z. Bsp. $n=3$\\ 
\begin{minipage}[c]{0.5\textwidth}
	\begin{tikzpicture}[scale=0.5]
	\node at (-8,4) {$\mathds{Z}$};
	\draw  (-3,4) ellipse (3.5 and 5);
	\draw (-6,6) -- (0,6);
	\draw (-6,1.5) -- (0,1.5);
	\node at (-5,1) {0};
	\node at (-4,1) {$\pm$3};
	\node at (-3.5,0) {$\pm$6};
	\node at (-2.5,0.5) {$\pm$9};
	\node at (-1.5,0.5) {...};
	\node at (-5.5,5) {1};
	\node at (-4,5) {4};
	\node at (-2.5,5) {7};
	\node at (-5.5,3) {-2};
	\node at (-4,3) {-5};
	\node at (-2.5,3) {-8};
	\node at (-0.5,4) {...};
	\node at (-4.5,8) {2};
	\node at (-3,8) {5};
	\node at (-1.5,8) {8};
	\node at (-5,6.5) {-1};
	\node at (-3.5,6.5) {-4};
	\node at (-2,6.5) {-7};
	\node at (-1,7.5) {...};
	\draw (1.2588,6.1319) arc (-20.0001:20:4);
	\draw (0.5642,1.4288) arc (-40.0003:20:4);
	\draw (0.8492,0.8551) arc (20.0013:-20:2.5);
	\node at (2.75,7.5) {Rest 2};
	\node at (2.75,3.5) {Rest 1};
	\node at (2.75,0) {Rest 0};
	\end{tikzpicture}
\end{minipage}
\begin{minipage}[c]{0.5\textwidth}
	bei Division durch 3
\end{minipage}\\
$\mathbb{Z}_3 = \{\underbrace{0,\qquad1,\qquad2}_{\textrm{Restklassen, Repräsentanten}}\}$ \\
\begin{itemize}
	\item[a)] $(\mathbb{Z}_n, \oplus)$ mit $a \oplus b = a+b \mod n$. Z.B. in $\mathbb{Z}_3$ ist $2 \oplus 1 = 0$ \\
	\\
	\uline{$(\mathbb{Z}_n, \oplus)$ ist abelsche Gruppe}:
	\begin{itemize}
		\item abgeschlossen: $a+b \mod n \in \{0,...,n-1\}$
		\item assoziativ: $a + (b + c) \mod n = (a + b) + c \mod n$
		\item Neutralelement: $a + 0 \equiv 0  + a \equiv a\quad\pmod{n}$
		\item Inverse zu $ a \in \mathbb{Z}_n$: Für welches $b \in \mathbb{Z}_n$ ist $a+b \mod n = 0$ ?\\
		Wähle $b$ so, dass $a+b = n$, falls $a \neq 0$ (sonst b = 0) \\
		z.B. in $\mathbb{Z}_3: a = 1 \Rightarrow b = 2,\quad a = 2 \Rightarrow b = 1,\quad a = 0, b = 0$
		\item kommutativ: $a+b \mod n = b +a \mod n$
	\end{itemize}
\item[b)] $(\mathbb{Z}_n, \odot)$ mit $a \odot b = ab \mod n$ \\
Ist i.A. keine Gruppe:
\begin{itemize}
	\item assoziativ $\checkmark$
	\item Neutralelement: $e = 1$ $\checkmark$
	\item Aber: 0 hat keine Inverse! Es gibt kein $a \in \mathbb{Z}_n\colon \underbrace{0 \cdot a \mod n}_{0} = 1$ (\Lightning) \\
	Hat $z \neq 0$ eine Inverse bzgl. $\odot$? \\
	\\
	$\bar{z}$ invers zu $z$, wenn $\bar{z} \cdot z \equiv 1\pmod{n}$ \\
	z.B. in $\mathbb{Z}_15$ gilt: 
	\begin{itemize}
		\item $2 \cdot 8 = \quad16 \equiv 1 \pmod{15}$, d.h. 2 und 8 sind zueinander invers
		\item Alle Vielfachen von 5 haben Rest $0,5,10$, d.h. \\
		$k\cdot 5 \mod 15 \in \{0,5,10 \} \quad\forall k \in \mathbb{Z} \Rightarrow 5$ hat kein Inverses
	\end{itemize}
	Allgemein: 
	\begin{align*}
	z \text{ invertierbar } &\Leftrightarrow \exists \bar{z} \in  \mathbb{Z}_n: z \odot \bar{z} = 1 \\
	& \Leftrightarrow \exists \bar{z} \in \mathbb{Z}_n\quad \exists q \in \mathbb{Z}: \bar{z} \cdot z = qn +1 \\
	&\Leftrightarrow \exists \bar{z},q \in \mathbb{Z}: \bar{z} \cdot z - qn = 1 \\
	&\overset{*}{\Leftrightarrow} \text{ggT}(z,n)= 1
	\end{align*}
\end{itemize} 
\end{itemize}
\subsubsection*{Beweis von *}
\begin{itemize}
	\item['$\Leftarrow$'] Lemma von Bézout/Erweiterter Euklidischer Algorithmus (EEA): \\
	$a,b \in \mathbb{Z} \Rightarrow \exists s,t \in \mathbb{Z}: \text{ggT}(a,b)= s \cdot a + t \cdot b$\\
	Hier: $a = z,\quad b = n,\quad s = \bar{z},\quad t = -q$
	\item['$\Rightarrow$'] Übung (Übungsblatt 5, A1c)
\end{itemize}
Also: Nur die zu $n$ teilerfremden Zahlen in $\mathbb{Z}_n$ haben Inverse. Z.B.: In $\mathbb{Z}_{15}$ sind $1,2,4,7,8,11,13,14$ bzgl. $\odot$ invertierbar. \\
\uline{Bezeichnung}: $\mathbb{Z}^*_n = \{z \in \mathbb{Z}_n \mid \text{ggT}(z,n)= 1\}$ ist Gruppe mit Ordnung $| \mathbb{Z}^*_n | = \phi(n)$\\
 (Eulersche $\phi-$Funktion, $\phi(n)$ ist Anzahl der zu $n$ teilerfremden Zahlen zwischen 1 und $n$).\\
 \\
Berechnung der Inversen in $\mathbb{Z}^*_n$: 
\begin{align*}		
\text{EEA}:\qquad z \in \mathbb{Z}^*_n &\Rightarrow \exists s,t \in \mathbb{Z}: sz + tn = 1 \\
&\Rightarrow s \cdot z \equiv 1 \pmod{n}\\
& \Rightarrow s \text{ invers zu } z
\end{align*}
\newpage
\subsection{Satz (Eigenschaften von Gruppen)} 
$G$ Gruppe.
\begin{itemize}
	\item[i)] Das Neutralelement von $G$ ist eindeutig. 
	\item[ii)] Die Inverse zu jedem $a \in G$ ist eindeutig.
	\item[iii)] $a,b \in G \Rightarrow (ab)^{-1} = b^{-1} \cdot a^{-1}$
\end{itemize}
\subsubsection*{Beweis}
\begin{itemize}
	\item[i)] Angenommen $e_1,e_2$ Neutralelemente \\
	$\Rightarrow e_1 = e_1 \cdot e_2 = e_2$
	\item[ii)] Angenommen $a \in G$ hat 2 Inversen $x,y$\\
	 $x,y \in G \Rightarrow x = x\underbrace{(ay)}_{e} = \underbrace{(xa)}_e y = y$
	\item[iii)] 
	\begin{itemize}
		\item[$\ast$] $(ab)^{-1} \cdot (ab) \underset{\textrm{Vor.}}{=} (b^{-1}a^{-1})(ab) = b^{-1}\underbrace{(a^{-1}a)}_e b = \underbrace{b^{-1}b}_e = e$
		\item[$\ast$] $(ab)(ab^{-1})$ analog
	\end{itemize}
\hfill$\square$
\end{itemize}
\subsection{Satz (Gleichungen lösen in Gruppen)}
$G$ Gruppe, $a,b \in G$
\begin{itemize}
	\item[i)] $\exists! x \in G: a \cdot x = b$. \quad Es ist $x = a^{-1} \cdot b$
	\item[ii)] $\exists! y \in G: y \cdot a = b$. \quad Es ist $y = b \cdot a^{-1}$
	\item[iii)] $ax = bx$ für ein $x \in G \Rightarrow a = b$\\
	bzw. $ya = yb$ für ein $y \in G \Rightarrow a = b$ (Kürzungsregel)
\end{itemize}
\subsubsection*{Beweis}
\begin{itemize}
	\item[i)] $x = a^{-1}b$ erfüllt $ax = a(a^{-1}b) = \underbrace{(aa^{-1})}_e b = b$
	\item[ii)] Analog zu i)
	\item[iii)] $a = a\underbrace{(xx^{-1})}_{e} = (ax)x^{-1} = (bx)x^{-1} = b\underbrace{(xx^{-1})}_{e} = b$ 
\end{itemize}
\hfill$\square$
\subsubsection*{Untergruppen und Nebenklassen}
\subsection{Definition (Untergruppe)}
$(G, \cdot)$ Gruppe, $\emptyset \neq U \subseteq G$.\\
 $U$ heißt \uline{Untergruppe} von $G$ $(U \leq G)$, wenn $U$ bzgl. '$\cdot$' eine Gruppe ist.
\end{document}
