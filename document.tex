\documentclass[12pt,titlepage]{article}

%Packages%

\usepackage[ngerman]{babel}
\usepackage[utf8]{inputenc}
\usepackage{amssymb}
\usepackage{amsmath}
\usepackage{dsfont}
\usepackage{mathtools}
\usepackage[pd,col,3d]{mgtex_24}
\usepackage{tikz}
\usepackage{marvosym}

%TitlePage%

\begin{titlepage}
	\title{Mathematik III}
	\date{18.10.2016}
\end{titlepage}

%NewCommands%

\newcommand{\R}{\mathds{R}}
\newcommand{\N}{\mathds{N}}
\newcommand{\uline}[1]{\underline{#1}}
\newcount\colveccount
\newcommand*\colvec[1]{
	\global\colveccount#1
	\begin{pmatrix}
		\colvecnext
	}
	\def\colvecnext#1{
		#1
		\global\advance\colveccount-1
		\ifnum\colveccount>0
		\\
		\expandafter\colvecnext
		\else
	\end{pmatrix}
	\fi
}
\newcommand{\vecspace}[2]{\langle#1\rangle_{#2}}
\newcommand{\vecspaceR}[1]{\vecspace{#1}{\R}}

%ReNewCommands%

\renewcommand{\>}{\rightarrow}
\renewcommand{\*}{\cdot}
\renewcommand{\O}{\mathcal{O}}
\renewenvironment{rcases}{% 
	\left.\renewcommand*\lbrace.% 
	\begin{cases}}% 
	{\end{cases}\right\rbrace}
\renewcommand{\vec}[1]{\colvec{#1}}

%Document%

\begin{document}
	\maketitle
	\tableofcontents
	\newpage
	\section{Vektorräume}
	\small{Bemerkung: 1.1-1.10 identisch mit 8.1-8.10 aus Mathematik 2, SS16}
	\subsection{Definition (Reelle Vektorräume)}
	Ein \underline{$\R$-Vektorraum V} ist eine nichtleere Menge, deren Elemente \underline{Vektoren} genannt werden (Bezeichnung mittels kleiner lateinischer Buchstaben, $v,w,x,y,...$), auf der eine Addition $+$ definiert ist, $+\colon V\times V\>V$; und eine Multiplikation mit reellen Zahlen ('Skalare') (Bezeichnung mittels kleiner griechischer Buchstaben $\alpha, \beta, \gamma, \lambda,\mu,...$), $\*\colon\R\times V\>V$, so dass gilt:
	\begin{itemize}
		\item[(1.1)] $u+v+w=u+(v+w)\qquad\forall u,v,w\in V$
	   	\item[(1.2)] Es existiert ein Vektor $\O\in V$ ('\underline{Nullvektor}') mit $v+\O=\O+v=v\qquad\forall v\in V$
	   	\item[(1.3)] Zu jedem $v\in V$ existiert ein Vektor $-v\in V$ mit $v+(-v)=\O$
		\item[(1.4)] $u+v=v+u\qquad\forall u,v\in V$
	\end{itemize}
	(Diese Eigenschaften (1.1) bis (1.4) kann man zusammenfassen als '$(V,+)$ ist eine kommutative Gruppe').
	\begin{itemize}
		\item[(2.1)] $\overset{\textrm{Addition in }\R}{(\lambda+\mu)}\*v=\lambda\*v\overset{\textrm{Addition in }V}{+}\mu\*v\qquad\forall\lambda,\mu\in\R,v\in V$
		\item[(2.2)] $\lambda(v+w)=\lambda v+\lambda w\qquad\forall\lambda\in\R,v,w\in V$
		\item[(2.3)] $\overset{\textrm{Multiplikation in }\R}{(\lambda\*\mu)}\*v=\lambda\*\overset{\textrm{Multiplikation mit Skalar}}{(\mu\*v)}\qquad\forall\lambda,\mu\in\R,v\in V$
		\item[(2.4)] $1\*v=v\qquad\forall v\in V$
	\end{itemize}
	\subsection{Beispiel}
	\begin{itemize}
		\item[a)] trivialer Vektorraum Nullraum: $V=\{\O\}$\\
		Es gilt $\O+\O\coloneqq\O,\quad\lambda\*\O\coloneqq\O\qquad\forall\lambda\in\R$
		\item[b)] $V=\R^n$, Raum aller 'Spaltenvektoren' der Länge $n$ über $\R$, Elemente haben die Form $\vec3{x_1}{...}{x_n}$ mit $x_1,...,x_n\in\R$.\\
		$\O=\vec3{0}{...}{0},\quad\vec3{x_1}{...}{x_n}+\vec3{y_1}{...}{y_n}=\vec3{x_1+y_1}{...}{x_n+y_n},\quad\lambda\*\vec3{x_1}{...}{x_n}=\vec3{\lambda\*x_1}{...}{\lambda\*x_n}$
		\item[c)] $\R$ ist ein $\R$-Vektorraum.\\
		Vektoren: reelle Zahlen.\\
		Skalare: reelle Zahlen.\\
		$\O=0$
		\item[d)] Funktionenraum:\\
		$M\neq\emptyset$ Menge. $V=\mathcal{F}(M,\R)\coloneqq\{f\colon M\>\R\}$\\
		Menge der auf $M$ definierten reellen Funktionen.\\
		Für $f,g\in V,\quad\lambda\in\R$ sei
		\begin{itemize}
			\item $f+g\colon M\>\R,\quad(f+g)(x)=f(x)+g(x)\quad\forall x\in M$
			\item $\lambda\*f\colon M\>\R,\quad(\lambda\*f)(x)=\lambda\*f(x)\quad\forall x\in M$
		\end{itemize}
		Dann ist $V$ mit $\R,+,\*$ ein Vektorraum. Nullvektor ist $f=0\colon M\>\R,\quad f(x)=0\qquad\forall x\in M$.\\
		(kurz: $f\equiv0$, identisch Null)
	\end{itemize}
	\subsection{Lemma}
	Sei $V$ ein $\R$-Vektorraum, $v\in V,\quad\lambda\in\R$
	\begin{itemize}
		\item[a)] $0\*v=\O$
		\item[b)] $\lambda\*\O=\O$
		\item[c)] Zu jedem $v\in V$ ist der Vektor $-v$ aus (1.3) in 8.1 eindeutig bestimmt.
		\item[d)] $(-1)\*v=-v$
	\end{itemize}
	\subsubsection*{Beweis}
	\begin{itemize}
		\item[a)] \begin{align*}
		\O\overset{\textrm{(1.3)}}{=}\overbrace{\underbracket{0\*v}_{}}^{x}+\overbrace{(-0\*v)}^{-x}&=\underbracket{(0+0)v}_{}+(-0\*v)\\
		&\overset{\textrm{(2.1)}}{=}(0\*v+0\*v)+(-0\*v)\\
		&\overset{\textrm{(1.1)}}{=}0\*v+(0*v+(-0\*v))\\
		&\overset{\textrm{(1.3)}}{=}0\*v+\O\\
		&\overset{\textrm{(1.2)}}{=}0\*v
		\end{align*}
		\item[b)] Wie a), starte mit $\O=\lambda\*\O+(-\lambda\*\O)$, erhalte $\O=\lambda\*\O$
		\item[d)] \begin{align*}
		\underbracket{v+(-1\*v)}_{}&=1\*v+(-1\*v)\\
		&\overset{\textrm{(2.1)}}{=}(1+(-1))v\\
		&=0\*v\\
		&\overset{\textrm{a)}}{=}\O\\
		&\overset{\textrm{(1.3)}}{=}v+(-v)
		\end{align*}
		Addiere auf beiden Seiten $-v$:
		\begin{align*}
		\underbracket{v+(-1)v}_{}+(-v)&=v+(-v)+(-v)\\
		&\Rightarrow-1\*v=-v		
		\end{align*}
		\item[c)] Angenommen, zu $v\in V$ gibt es $-v$ und $-v^\prime$ mit $v+(-v)=\O$ und $v+(-v^\prime)=\O$. Dann ist $v+(-v)=v+(-v^\prime)\overset{+(-v)\textrm{auf beiden Seiten}}{\Rightarrow}-v=-v^\prime$
	\end{itemize}
	\hfill$\square$
	\subsection{Definition}
	Sei $V$ ein $\R$-Vektorraum.\\
	Eine Teilmenge $U\subseteq V,\quad U\neq\emptyset$ heißt \underline{Unter(vektor)raum von $V$}, falls $U$ bezüglich der Addition auf $V$ und der Multiplikation mit Skalaren selbst ein Vektorraum ist.
	\subsection{Beispiel}
	\begin{itemize}
		\item[a)] $V=\R^2,\quad U=\{\vec2{0}{0}\}$ ist Unterraum von $V$
		\item[b)] $V=\R^2,\quad U=\{\vec2{1}{2}\}$ ist kein Unterraum von $V$, z.B. (1.2) ist verletzt, Addition funktioniert auch nicht: $\vec2{1}{2}+\vec2{1}{2}=\vec2{2}{4}\notin U$
		\item[c)] $V=\R^2,\quad U=\{\vec2{\lambda}{0}|\lambda\in\R\}$ ist ein Unterraum von $V$ (prüfe alle Eigenschaften von Definition 8.1) $\>$ umständlich, einfacher geht es mit 8.6
	\end{itemize}
	\subsection{Satz (Unterraumkriterium)}
	Sei $V$ ein $\R$-Vektorraum, sei $\emptyset\neq U\subseteq V$.\\
	Dann ist $U$ Unterraum von $V$ genau dann, wenn gilt ($\Leftrightarrow$):
	\begin{itemize}
		\item[(1)] $v\in U,\quad\lambda\in\R\Rightarrow\lambda\*v\in U$
		\item[(2)] $v,w\in U\Rightarrow v+w\in U$
	\end{itemize}
	(oder äquivalent: $\forall v,w\in U, \forall\lambda,\mu\in\R$ ist $\lambda\*v+\mu\*w\in U$)\\
	Man sagt: $U$ ist abgeschlossen bezüglich der Vektoraddition und der Multiplikation mit Skalaren.
	\subsubsection*{Beweis}
	\begin{itemize}
		\item[$\Rightarrow$] ist klar, da $U$ laut Definition 8.4 selbst Vektorraum
		\item[$\Leftarrow$] rechne die Vektorraumaxiome nach (Definition 8.1, also z.B. $\O\in U$,...)
	\end{itemize}
	\hfill$\square$
	\subsection{Beispiel}
	\begin{itemize}
		\item[a)] $\quad$\\
		\begin{minipage}[c]{0.5\textwidth}
			$V$ ist ein $\R$-Vektorraum, $\O\neq v\in V$.\\
			Dann ist $G=\{\lambda\*v|\lambda\in\R\}$ ein Unterraum.\\
			$V=\R^2,\R^3$: $G$ ist Gerade durch Nullpunkt (geometrisch), z.B.\\ $v=\vec2{2}{1},w=\vec2{1}{2}$\\
			Aber: $G^\prime=\{w+\lambda\*v|\lambda\in\R,\quad w\in V\}$ ist kein Unterraum für $w\neq \mu\*v,\quad \mu\in\R$. Warum? Z.B. $\O\notin G^\prime$
		\end{minipage}
		\begin{minipage}[c]{0.25\textwidth}$\quad$
			\begin{tikzpicture}[scale=0.5]
			\draw[->]
			(-0.2,0) -- (5,0) node[right] {$x$};
			\draw[->]
			(0,-0.5) -- (0,5) node[above] {$f(x)$};
			\draw[color=blue, thick, ->]
			(0,0)--(2,1)node[below]{$v$};
			\draw[color=blue, thick, ->]
			(0,0)--(1,2)node[above]{$w$};
			\draw[color=blue, ->]
			(0,0)--(4,2)node[below]{$\quad2\*v$, $G$};
			\draw[color=blue, ->]
			(0,0)--(2,4)node[above]{$2\*w$};
			\draw[color=blue, ->]
			(0,1.5)--(3,3)node[right]{$G^\prime$};
			\end{tikzpicture}
		\end{minipage}
		\item[b)] $V=\R^3,\qquad U_1=\{\vec3{x_1}{x_2}{x_3}\in\R^3|x_1+x_2-x_3=0\}$ ist Unterraum. Wir zeigen (1), (2) aus 8.6:
		\begin{itemize}
			\item $U_1\neq\emptyset$, z.B. $\O=\vec3{0}{0}{0}\in U_1$, denn $\overset{x_1}{0}+\overset{x_2}{0}-\overset{x_3}{0}=0$
			\item[(1)] Sei $\lambda\in\R,\quad v=\vec3{v_1}{v_2}{v_3}\in U_1$, d.h. $v_1+v_2-v_3=0$\\
			Prüfe: Ist $\lambda\*v\in U_1$? $\lambda\*v=\vec3{\lambda\*v_1}{\lambda\*v_2}{\lambda\*v_3}$
			\begin{align*}
			\lambda\*v_1+\lambda\*v_2-\lambda\*v_3&=\lambda(v_1+v_2-v_3)\\
			&=\lambda\*0\\
			&=0
			\end{align*}
			Also ist $\lambda\*v\in U_1$
			\item[(2)] Seien $v=\vec3{v_1}{v_2}{v_3},\quad w=\vec3{w_1}{w_2}{w_3}\in U_1$, d.h. $v_1+v_2-v_3=0,\quad w_1+w_2-w_3=0$. Gilt $v+w\in U_1$?  $v+w=\vec3{v_1+w_1}{v_2+w_2}{v_3+w_3}$
			\begin{align*}
			(v_1+w_1)+(v_2+w_2)-(v_3+w_3)&=\underbrace{(v_1+v_2-v_3)}_{=0}+\underbrace{(w_1+w_2-w_3)}_{=0}\\
			&=0
			\end{align*}
			Also $v+w\in U_1$
			\item Geometrische Interpretation:\\
			\begin{align*}
			U_1&=\{\vec3{x_1}{x_2}{x_1+x_2}|x_1,\quad x_2\in\R\}\\
			&=\{x_1\*\vec3{1}{0}{1}+x_2\*\vec3{0}{1}{1}|x_1,\quad x_2\in\R\}
			\end{align*}
			D.h. $U_1$ ist die Ebene durch $O=\vec3{0}{0}{0}$ mit den Richtungsvektoren $\vec3{1}{0}{1}$ und $\vec3{0}{1}{1}$
		\end{itemize}
		\item[c)] $U_2=\{\vec3{x_1}{x_2}{x_3}\in\R^3|x_1+x_2-x_3=1\}$ ist kein Unterraum. Z.B. $\vec3{0}{0}{0}=\O\notin U_2$: $0+0-0=0\neq1$.\\
		Anderes Argument: Sei $\lambda\in\R,\quad x=\vec3{x_1}{x_2}{x_3}\in U_2$, d.h. $x_1+x_2-x_3=1$. Gilt $\lambda\*x\in U_2$? $\lambda\*x=\vec3{\lambda x_1}{\lambda x_2}{\lambda x_3}$
		\begin{align*}
		\lambda x_1+\lambda x_2-\lambda x_3&=\lambda\underbrace{(x_1+x_2-x_3)}_{=1}\\
		&=\underbrace{\lambda=1}_{\textrm{nur für }\lambda=1}
		\end{align*}
		$\Rightarrow$ nicht erfüllt für $\lambda\neq1$.\\
		Geometrische Interpretation:\\
		\begin{align*}
		U_2&=\{\vec3{x_1}{x_2}{x_1+x_2-1}|x_1,\quad x_2\in\R\}\\
		&=\{\vec3{0}{0}{-1}+x_1\*\vec3{1}{0}{1}+x_2\*\vec3{0}{1}{1}|x_1,\quad x_2\in\R\}
		\end{align*}
		Ebene durch $\vec3{0}{0}{-1}$ mit Richtungsvektoren $\vec3{1}{0}{1}$ und $\vec3{0}{1}{1}$
		\item[d)] $U_3=\{\vec3{x_1}{x_2}{x_3}\in\R^3|x_1^2+x_2^2+x_3^2\leq1\}$ ist kein Unterraum, z.B.\\
		$\vec3{1}{0}{0}\in U_3,\qquad1^2+0^2+^2\leq1\quad\checkmark$, aber\\
		$2\*\vec3{1}{0}{0}=\vec3{2}{0}{0}\notin U_3$, denn $2^2+0^2+0^2\nleq1$\\
		Geometrische Interpretation:\\
		$U_3$ ist eine Kugel um $\vec3{0}{0}{0}$ mit Radius 1
		\item[e)] $I\subseteq\R$ Intervall\\Menge $C(I)$ ($C$: continuous, stetig) der stetigen Funktionen auf $I$ ist Unterraum von $\mathcal{F}(I,\R)$ (vgl. Beispiel 8.2d)).\\
		Menge der diffbaren Funktionen auf $I$ ist Unterraum von $C(I)$.
	\end{itemize}
	\subsection{Satz}
	$V$ ist ein $\R$.Vektorraum, $U_1,U_2$ sind Unterräume von $V$.
	\begin{itemize}
		\item[a)] $U_1\cap U_2=\{u\in V|u\in U_1\wedge u\in U_2\}$ ist Unterraum von $V$.
		\item[b)] $U_1+U_2\coloneqq\{u_1+u_2|u_1\in U_1\wedge u_2\in U_2\}$ \underline{Summe} von $U_1,U_2$ ist Unterraum von $V$\\
		(das ist nicht die Vereinigung $U_1\cap U_2$!)
	\end{itemize}
	\subsubsection*{Beweis}
	Prüfe Unterraumkriterium 8.6
	\begin{itemize}
		\item[a)] Übung: Prüfe $\O\in U_1\cap U_2$? $\checkmark$, (1), (2)
		\item[b)] \begin{itemize}
			\item $U_1+U_2\neq\emptyset$, denn $U_1+U_2\ni\O=\underbrace{\O}_{\in U_1}+\underbrace{\O}_{\in U_2}$
			\item Seien $v=u_1+u_2, \quad u_1\in U_1,\quad u_2\in U_2$ und\\
			$w=u_1^\prime+u_2^\prime,\quad u_1^\prime\in U_1,\quad u_2^\prime\in U_2$,\\
			also $v,w\in U_1+U_2$ und $\lambda,\mu\in\R$.\\
			\begin{align*}
			\Rightarrow\qquad\lambda v+\mu v&=\lambda(u_1+u_2)+\mu(u_1^\prime+u_2^\prime)\\
			&=\underbrace{\lambda u_1+\mu u_1^\prime}_{\in U_1}+\underbrace{\lambda u_2+\mu u_2^\prime}_{\in U_2}
			&\in U_1+U_2
			\end{align*}
		\end{itemize}
	\end{itemize}
	\subsection{Bemerkung}
	\begin{itemize}
		\item[a)] lässt sich für unendlich viele Unterräume ausweiten
		\item[b)] lässt sich für endlich viele Unterräume ausweiten
		\item $U_1\cup U_2$ ist im Allgemeinen \underline{kein} Unterraum
	\end{itemize}
	\subsection{Beispiel}
	\begin{itemize}
		\item $v=\vec2{1}{0}\in\R^2\qquad\qquad G_1=\{\lambda v|\lambda\in\R\}$
		\item $w=\vec2{2}{1}\in\R^2\qquad\qquad G_2=\{\mu w|\mu\in\R\}$
	\end{itemize}
	(vgl. 8.7a), Geraden durch $\vec2{0}{0}$, Unterräume
	\begin{itemize}
		\item $G_1+G_2$ ist Ebene
		\item $G_1\cap G_2$ ist $\O=\vec2{0}{0}$
	\end{itemize}

%MATHEMATIK 3%%MATHEMATIK 3%%MATHEMATIK 3%%MATHEMATIK 3%
%MATHEMATIK 3%%MATHEMATIK 3%%MATHEMATIK 3%%MATHEMATIK 3%
%MATHEMATIK 3%%MATHEMATIK 3%%MATHEMATIK 3%%MATHEMATIK 3%

	\subsection{Beispiel}
	\begin{minipage}[c]{0.5\textwidth}
	\InitGraph{5}{8}{1}{1}{0.7cm}
	\Viewpoint(300,70,10,8)[1]
	\SetCMYKColor(0.5,0,0,0)
	\ShowFullPlaneThrough(2.1,1,1)NormalTo(0,2,-2)(2.1)
	\Text[t]{$E$}
	\SetCMYKColor(0,0,0,255)
	\DDArrowAt(0,0,0)(6,0,0)
	\Text[b]{x}
	\DDArrowAt(0,0,0)(4,1,1)
	\SetDashed
	\DDArrowAt(0,0,0)(0,1000,0)
	\Text[b]{y}
	\SetNormal
	\DDArrowAt(0,30,0)(0,1000,0)
	\DDArrowAt(0,0,0)(0,0,6)
	\Text[l]{z}
	\SetCMYKColor(0,1,1,0)
	\DDArrowAt(0,0,0)(2,0,0)
	\Text[b]{$v$}
	\SetCMYKColor(1,1,0,0)
	\DDArrowAt(0,0,0)(0,1,1)
	\Text[t]{$u$}
	\Text[t]{$u+2v$}
	\CloseGraph
	\end{minipage}
	\begin{minipage}[c]{0.5\textwidth}
		\begin{itemize}
			\item $ u = \vec3{0}{1}{1}$
			\item $ v = \vec3{2}{0}{0}$
			\item $ E = \{ \lambda_1 \cdot \vec3{0}{1}{1} + \lambda_2 \cdot \vec3{2}{0}{0} | \lambda_1, \lambda_2 \in \R\}$
		\end{itemize}
	\end{minipage}
\\
\\
\\
\begin{itemize}
	\item E $\subseteq \R^3 $ ist Untervektorraum (UVR) und wird \uline{aufgespannt/erzeugt} von $u$ und $v$. Man nennt $\{\vec3{0}{1}{1},\vec3{2}{0}{0}\}$ \uline{Erzeugendensystem} von $E$.
	\item D.h. $w \in E \Leftrightarrow \exists \lambda_1, \lambda_2 \in \R: w = \underbrace{\lambda_1 \cdot u + \lambda_2 \cdot v}_{\text{Linearkombination von $u$ und $v$}}$
	\item $w \notin E$, z.B. $w = \vec3{0}{0}{1}$ ergibt: \\
	$\vec3{0}{0}{1} = \lambda_1 \cdot u + \lambda_2 \cdot v = \lambda_1 \vec3{0}{1}{1} + \lambda_2 \vec3{2}{0}{0} \\
	\\
	 \Rightarrow \begin{rcases}
	 	\textrm{Letzte Zeile: }1=\lambda_1\\
	 	\textrm{Zweite Zeile: }0=\lambda_1
	 \end{rcases}$\Lightning
 \\
	$\Rightarrow \vec3{0}{0}{1} \notin E$
\end{itemize}
\subsection{Definition (Linearkombination, Erzeugendensystem)}
$V: \R$-VR (V ist Vektorraum in den reellen Zahlen) \\
\begin{itemize}
	\item[(i)] $v_1, ... , v_m \in V$ und $\lambda_1,...,\lambda_m \in \R$\\ Der Vektor $\lambda_1 \* v_1 + ... + \lambda_m \* v_m$ heißt \uline{Linearkombination} von $v_1,...,v_m$.
	\item[(ii)] Sei $M \subseteq V$. Dann ist
	\begin{center}
		$\langle M \rangle_{\R} = \{ \sum_{k = 1}^{n} \lambda_k \cdot v_k \vert \lambda_k \in \R, v_k \in M, n \in \mathbb{N}\}$
	\end{center}
	der \underline{von $M$ aufgespannte/erzeugte UVR} von V \\
	\\
	Vereinbarung: $\langle \emptyset \rangle = \{0\}$\\
	Schreibweise: $M = \{v_1,...,v_m\}$\\
	\noindent\hspace*{22mm}$\langle M \rangle_{\R} = \langle v_1,..., v_m\rangle_{\R} $
	\item[(iii)]
	Ist $V = \langle M \rangle_{\R}$, so heißt $M$ ein \uline{Erzeugendensystem} von $V$. $V$ heißt \uline{endlich erzeugt}, falls es ein endliches Erzeugendensystem gibt.
\end{itemize}
\subsection{Bemerkung}
$M \subseteq V \Rightarrow \langle M \rangle_{\R}$ ist der kleinste UVR von $V$, der $M$ enthält.\\
\subsubsection*{Beweis}
\begin{itemize}
	\item $\langle M \rangle_{\R}$ ist UVR. erfüllt Kriterien von 1.6, daher klar: \\
	1.6 2) erfüllt. $u \in \langle M \rangle_{\R} \Rightarrow u = \lambda_1 \cdot v_1 + ... + \lambda_n \cdot v_n\quad(M = \{v_1, ..., v_n \})\\ \Rightarrow \lambda \cdot u = \underbrace{\lambda  \lambda_1}_{\in \R} \cdot v_1 + ... + \underbrace{\lambda \lambda_n}_{\in \R} \cdot v_n$\\
	1.6 3) ähnlich.
	\item Angenommen $U$ ist der kleinste UVR, so dass $M \subseteq U$. \\
	Z. z.: $\langle M \rangle_{\R} = U.$\\
	Wegen 1.6 enthält $U$ alle Linearkombinationen von Vektoren aus M. \\
	$\Rightarrow \langle M \rangle_{\R} \subseteq U \Rightarrow U$ kann nicht kleiner sein als $\langle M \rangle_{\R} \Rightarrow \langle M \rangle_{\R} = U$\hfill$\square$
\end{itemize}
\subsection*{Fortsetzung Bsp. 1.11}
\begin{itemize}
	\item[a)] $E = \langle \vec3{0}{1}{1}, \vec3{2}{0}{0} \rangle_{\R}$
	\item[b)] $\R^n$ wird erzeugt von $e_j = \vec5{0}{\vdots}{1}{\vdots}{0}$, wobei j die Stelle ist, an der der Vektor 1 ist. \\
	$R^n = \vecspaceR{\vec4{1}{0}{0}{\vdots}, \vec4{0}{1}{0}{\vdots}, ...,\vec4{0}{0}{\vdots}{1}}$ "kanonische Einheitsvektoren" \\
	$v = \vec3{v_1}{\vdots}{v_n} = v_1 \cdot e_1 + v_2 \cdot e_2 + ... + e_n \cdot v_n$
	\item[c)] Spannen $\vec2{1}{1}$ und $\vec2{1}{2}$ den $\R^2$ auf? \\
	Wenn ja, dann muss für $\vec2{x}{y} \in  \R^2\qquad \alpha, \beta \in \R$ existieren mit 
	\begin{align*}
	&&\alpha \cdot \vec2{1}{1} + \beta \cdot \vec2{1}{2} &= \vec2{x}{y}\\
	\Leftrightarrow&& \alpha + \beta &= x \\
	&&\alpha + 2\beta &= y \\
	\Rightarrow &&\alpha &= x - \beta\\
	&& & = y - 2 \beta \\
	\Leftrightarrow &&\beta &= y - x\\
	&&\alpha &= 2x -y 
	\end{align*}
	$\Rightarrow$\quad Allg. $ \vec2{x}{y} = (2x-y) \cdot \vec2{1}{1} + (y-x)\cdot \vec2{1}{2} \Rightarrow \R^2 = \langle \vec2{1}{1}, \vec2{1}{2} \rangle_{\R}$
	\item[d)] Spannen $\vec2{1}{2}$ und $\vec2{3}{6}$ den $\R^2$ auf? \\
	Nein, denn $\vec2{3}{6}$ ist $3 \cdot \vec2{1}{2} \Rightarrow \langle\vec2{1}{2}, \vec2{3}{6} \rangle_{\R} =  \langle \vec2{1}{2} \rangle_{\R} 
	= \{\lambda \cdot \vec2{1}{2}\vert \lambda \in \R \} \subsetneq \R^2$
	\item[e)]
	$\langle \vec2{0}{1}, \vec2{0}{1} \rangle_{\R} = \langle \vec2{1}{1}, \vec2{1}{2} \rangle_{\R} = \R^2$, d.h. Erzeugendensysteme sind \uline{nicht} eindeutig!
	\item[f)] 
	$\langle \vec2{1}{1}, \vec2{1}{2}, \vec2{2}{3} \rangle_{\R} = \langle \vec2{1}{1}, \vec2{1}{2} \rangle_{\R}$, da $\vec2{2}{3} = 
	\vec2{1}{1} + \vec2{1}{2}$.\\
	D.h. $M = \{\vec2{1}{1}, \vec2{1}{2}, \vec2{2}{3} \}$ ist kein \uline{minimales} Erzeugendensystem des $\R^2$, denn $v \in M$ kann immer dargestellt werden als Linearkombination von Vektoren aus $M \setminus {v}$. \\
	Man sagt: $\vec2{1}{1}, \vec2{1}{2}, \vec2{2}{3}$ sind \uline{linear abhängig}.
\end{itemize}
\subsection*{Ergänzung zu 1.13}
	Bsp: $M = \{\vec3{0}{1}{1} \} \Rightarrow \vecspaceR{M} = \{ \lambda \vec3{0}{1}{1} \vert  \lambda  \in  \R \}$ Gerade 
	\begin{itemize}
		\item $\vecspaceR{M} \supseteq M$
		\item $ E = \{\lambda_1 \vec3{0}{1}{1}  + \lambda_2 \vec3{2}{0}{0} \vert \lambda_1, \lambda_2 \in \R \} \supseteq M $
	\end{itemize}
	$\vecspaceR{M}$ Gerade, E Ebene, d.h. E ist größer als $\vecspaceR{M}$\\
	$\vecspaceR{M}$ ist der kleinste UVR von $\R^3$, der M enthält.
	\subsection{Definition: Lineare Unabhängigkeit}	
	\begin{itemize}
		\item $V\colon\quad\R - VR,\quad v_1,...,v_n$ heißen \uline{linear unabhängig}, wenn gilt: 
		\begin{center}
			$\begin{rcases}
			\lambda_1 \cdot v_1 + ... + \lambda_m \cdot v_m = 0 \\
			\lambda_1,...,\lambda_m\in\R
			\end{rcases}
			\Rightarrow \underbrace{\lambda = \lambda_2 = ... = \lambda_m = 0}_{\textrm{einzige Lösung!}}$
		\end{center}
		\item $M \subseteq V$ heißt linear unabhängig, wenn gilt: \\
		Für beliebiges $m \in \N $ und $v_1,...,v_m \in M$ paarweise verschieden sind $v_1,...,v_m$ linear unabhängig
		\item Ist in obigen beiden Fällen (mindestens) $\lambda_i \neq 0$, dann sind die Vektoren linear abhängig
	\end{itemize}
\subsection{Beispiel}
\begin{itemize}
	\item[a)] $\mathcal{O}$ ist linear abhängig, da $\lambda \cdot \mathcal{O} = 0 \qquad \forall \lambda \neq 0$
	\item[b)] Sind $\vec2{1}{2} , \vec2{-3}{1}, \vec2{1}{-5}$ linear abhängig in $\R^2$ ? \\
	$\lambda_1 \cdot \vec2{1}{2} + \lambda_2 \cdot \vec2{-3}{1} + \lambda_3 \cdot \vec2{1}{-5} = \mathcal{O}$\\
	$\begin{cases}
	\text{I}  \qquad \lambda_1 -3\lambda_2 + \lambda_3  &= 0 \\
	\text{II} \qquad 2 \lambda_1 + \lambda_2 - 5 \lambda_3 &= 0
	\end{cases}$\quad
	Erfüllt für $\lambda_1 = \lambda_2 = \lambda_3 = 0$. Aber hier gibt es noch die Lösung: $\lambda_1 = 2,\quad \lambda_2 = \lambda_3 = 1$!\\
	$\Rightarrow$ Vektoren sind linear abhängig 
	\item[c)] 
	$\vec3{1}{0}{0}, \vec3{0}{1}{0}, \vec3{0}{0}{1}$ linear unabhängig (l.u.) in $\R^3$ 
	\item[d)]
	$v \neq \mathcal{O},\quad v \in V,\quad v$, ist linear unabhängig \\
	Angenommen es existiert $\lambda \neq 0$ mit $\lambda \cdot v = 0$. \\
	$\Rightarrow v = (\frac{1}{\lambda} \cdot \lambda)\* v = \frac{1}{\lambda} \cdot (\lambda \cdot v) = \mathcal{O}$ \Lightning
	\item[e)]
	\begin{align*}
	v,w \text{ linear abhängig } &\Leftrightarrow v = \lambda w \text{ , für ein } \lambda \in \R \\
	&\Leftrightarrow v \in \vecspaceR{w}
	\end{align*}
	\item[f)]In $V = \mathcal{F}(\R{, \R}) = \{ f: \R \rightarrow \R \vert \text{ f Abbildung} \} $ sind die Vektoren
	\begin{itemize}
		\item $f(x) = x,\quad g(x) = x^2 $ linear unabhängig
		\item $f(x) = \sin^2(x),\quad g(x) = \cos^2(x), \quad h(x) = 2$ linear abhängig: \\
		\begin{align*}
		2&=2\*(\sin^2x+\cos^2x)\\
		&=2\sin^2x+2\cos^2x\\
		0&=\underbrace{2}_{\lambda_1}\sin^2x+\underbrace{2}_{\lambda_2}\cos^2x\underbrace{-1}_{\lambda_3}\*2
		\end{align*}
	\end{itemize}
\end{itemize}
	\subsection{Satz}
$M = \{v_1,...,v_n \} \subseteq V$\\
\begin{itemize}
	\item[(i)] $M$ linear unabhängig $\Leftrightarrow  \text{ Zu jedem } v \in \vecspaceR{M}$ gibt es eindeutig bestimmte \\
	\noindent\hspace*{43mm}$\lambda_1, ... \lambda_n  \in \R:v = \sum_{i= 1}^{n} \lambda_i \cdot v_i$
	\item[(ii)] $M$ linear unabhängig, $v \notin \vecspaceR{M} \Rightarrow M \cup \{ v \} $ linear unabhängig 
\end{itemize}
\subsubsection*{Beweis}
\begin{itemize}
	\item[(i)] 
	\begin{itemize}
		\item[($\Leftarrow$)] $ \mathcal{O} \in \vecspaceR{M} \Rightarrow \exists$ eindeutig bestimmte $\lambda_1, ... , \lambda_m \in \R:$\\
		\noindent\hspace*{23mm}$\mathcal{O} = \lambda_1 \cdot v_1 + ... + \lambda_n \cdot v_n$ \\
		Gleichung erfüllt für $\lambda_1 = ... = \lambda_n = 0$ (eindeutige Lösung)
		\item[($\Rightarrow$)] Sei $M$ linear unabhängig, $v \in \vecspaceR{M}$\\
		Angenommen $v = \sum_{i = 1}^{n} \lambda_i \cdot v_i = \sum_{i = 1}^{n} \mu_i \cdot v_i $\\
		\noindent\hspace*{25mm}$\Leftrightarrow  \sum_{i=1}^{n} \underbrace{(\lambda_i - \mu_i)}_{=0\textrm{, da }M\textrm{ linear unabhängig}} \cdot v_i = \mathcal{O}$\\
		\noindent\hspace*{25mm}$\Rightarrow \lambda_i = \mu_i \qquad \forall i = 1,...,n$
	\end{itemize}
	\item[(ii)] Z.z.: $\sum_{i=1}^{n} \lambda_i \cdot v_i + \lambda \cdot v = \mathcal{O} \Rightarrow \lambda_i = 0 \quad\forall i, \lambda = 0$ \\
	Annahme: $\lambda \neq 0 \Rightarrow v = \underbrace{-\frac{\lambda_1}{\lambda}}_{\in\R} \cdot v_1 - ... - \frac{\lambda_n}{\lambda} \cdot v_n $\\
	\noindent\hspace*{29mm}$\Rightarrow v \in \vecspaceR{M}$\Lightning. Also $\lambda = 0$ \\
	$\lambda_i = 0$, weil M linear unabhängig.\hfill$\square$
\end{itemize}
\subsection{Satz}
$M \subseteq V$ linear unabhängig genau dann, wenn gilt:
\begin{align*}
	N \subseteq M,\quad \vecspaceR{N} &= \vecspaceR{M} \Rightarrow N = M 
\end{align*}
In Worten: Man kann von $M$ keinen Vektor weglassen, ohne dass der von $M$ aufgespannte Raum sich verkleinert.  \\
\subsubsection*{Beweis}
\begin{itemize}
	\item[$(\Rightarrow)$] Sei $M \subseteq V$ linear unabhängig. \\
	Angenommen: Man kann doch aus $M$ Vektoren weglassen, d.h.
	\begin{align*}
		N\subseteq M,\quad\vecspaceR{N}=\vecspaceR{M}\textrm{ und }N\neq M
	\end{align*}
	\begin{align*}
		N\neq M&\Rightarrow\exists x\in M\setminus N&\textrm{(da }N\subseteq M\textrm{)}\\
		&\Rightarrow\exists v_1,...,v_n\in N&\textrm{paarweise verschieden und}\\
		&\qquad\exists\lambda_1,...,\lambda_n\in\R&\textrm{so dass}\\
		&\qquad x=\lambda_1v_1+...+\lambda_nv_n&\textrm{(da }\vecspaceR{N}=\vecspaceR{M}\textrm{)}\\
		&\Rightarrow\lambda_1v_1+...+\lambda_nv_n-x=\mathcal{O}\\
		&\qquad\underbrace{v_1,...,v_n}_{\in N},\quad \underbrace{x}_{\in M\setminus N}\textrm{ paarweise verschieden}
	\end{align*}
	Da $N\subseteq M$, ist $\underbrace{v_1,...,v_n,x}_{\textrm{linear abhängig}}\in M\Rightarrow M$ linear abhängig\Lightning\\
	Also muss $N=M$ gelten.
	\item[$(\Leftarrow)$] Sei $M$ linear abhängig. \\
	Z.z. Man kann Vektoren aus $M$ weglassen, d.h.:
	\begin{align*}
		\exists N \subseteq M,\quad \vecspaceR{N} = \vecspaceR{M}\textrm{ und }N \neq M
	\end{align*}
	$M$ linear abhängig $\Rightarrow \exists n \in \N\quad \exists v_1, ..., v_n \in M$\\
	\noindent\hspace*{37mm}$\exists \lambda_1,...,\lambda_n \in \R $ (mit $\lambda_i \neq 0$ für ein i) \\
	\noindent\hspace*{37mm}$\lambda_1 \cdot v_1 + ... + \lambda_n \cdot v_n = 0$\\
	\\
	O.B.d.A: $\lambda_1 \neq 0 \Rightarrow v_1 = -\frac{\lambda_2}{\lambda_1} \cdot v_2 - \frac{\lambda_3}{\lambda_1} \cdot v_3 - ... -\frac{\lambda_n}{\lambda_1} \cdot v_n$\\
	Setze $N = M \setminus \{v_1 \} \Rightarrow N \neq M$\\
	Da $v_1$ Linearkombination von $v_2,...,v_n$ folgt:\\
	 Jede Linearkombination von $v_1,...,v_n$ lässt sich ausdrücken als Linearkombination von $v_2,...,v_n \Rightarrow \vecspaceR{N} = \vecspaceR{M}$\hfill$\square$
\end{itemize}
\subsection*{Basis und Dimension}
Ein minimales Erzeugendensystem heißt Basis.
\subsection{Definition (Basis)}
$V$ \underline{endlich} erzeugter $\R$-VR. Eine endliche Menge $B\subseteq V$ heißt \underline{Basis}, falls
\begin{itemize}
	\item $\langle B\rangle_\R=V$ und
	\item $B$ linear unabhängig.
\end{itemize}
Für $V=\{\mathcal{O}\}$ ist $B=\emptyset$ die Basis.
\subsection{Beispiel}
\begin{itemize}
	\item[a)] $\{e_1,...,e_n\}$ ist Basis von $\R^n$ ('Standard-/kanonische Basis')
	\item[b)] Basisi ist nicht eindeutig.\\
	$B_1=\{\vec{2}{1}{0},\vec{2}{0}{1}\},\qquad B_2=\{\vec{2}{1}{1},\vec{2}{1}{2}$\\
	$\Rightarrow \langle B_1\rangle_\R=\langle B_2\rangle_\R$, da: $\vec{2}{1}{0}=2\vec{2}{1}{1}-\vec{2}{1}{2}$ und $\vec{2}{0}{1}=\vec{2}{1}{2}-\vec{2}{1}{1}$\\
	$\Rightarrow\vec{2}{1}{0},\vec{2}{0}{1}\in\langle B_2\rangle_\R\Rightarrow\R^2=\langle B_1\rangle_\R\subseteq\langle B_2\rangle_\R$
\end{itemize}
\subsection{Satz (Existenz von Basen)}
$V$ \underline{andlich} erzeugter $\R$-VR $\Rightarrow$ Jedes endliche Erzeugendensystem enthält Basis.
\subsubsection*{Beweis}
Sei $M\subseteq V$ endlich, $\langle M\rangle_\R=V$
\begin{itemize}
	\item $M$ linear unabhängig $\rightarrow$ fertig
	\item $M$ linear abhängig $\overset{1.17}{\Rightarrow}$ Man kann aus $M$ einen Vektor $v\in M$ weglassen,\\
	\noindent\hspace*{38mm} so dass $\langle M\setminus\{v\}\rangle_\R=V=\langle M\rangle_\R$. Nach endlich vielen\\
	\noindent\hspace*{38mm} Schritten liefert das Verfahren eine Basis.\hfill$\square$
\end{itemize}
\subsection*{Fragen}
\begin{itemize}
	\item Basis nicht eindeutig. Sind alle Basen gleich groß?
	\item geg. $w=\vec{3}{\frac{1}{3}}{0}{1}\in\R^3,\qquad S=\{e_1,e_2,e_3\}$. Wie kann man $w$ zu einer Basis ergänzen? Welche Vektoren aus $S$ sind geeignet?
	\begin{align*}
		w=\frac{1}{3}e_1+e_3&=\{\underbrace{w,e_1,e_3}_{\textrm{linear abhängig}}\}\textrm{ keine Basis, aber}\\
		&\quad\{\underbrace{w,e_1,e_2}_{\textrm{linear unabhängig}}\}\textrm{ Basis und }\{w,e_2,e_3\}\textrm{ Basis}
	\end{align*}
\end{itemize}
\subsection{Satz (Austauschlemma)}
$V$ endlich erzeugter $\R$-VR. Gegeben: $w\in V,\quad w\neq\mathcal{O},\quad w=\sum_{i=1}^{n}\lambda_iv_i$, wobei $B=\{v_1,...,v_n\}\subseteq V$ Basis von $V$.\\
$\Rightarrow\underbrace{(B\setminus\{v_j\})\cup\{w\}}_{(\star)}$ Basis, falls $\underbracket{\lambda_j\neq0}$
\subsubsection*{Beweis}
Z.z: ($\star$) ist Basis.
\begin{itemize}
	\item[1)] ($\star$) ist linear unabhängig.\\
	Z.z: \begin{align*}
		\sum_{i\neq j}^{}\mu_iv_i+\mu w=0&\Rightarrow\mu_i=0\textrm{ und }\mu=0\\
		\\
		\sum_{i\neq j}^{}\mu_iv_i+\mu w&=\sum_{i\neq j}^{}\mu_iv_i+\mu(\sum_{i=1}^{n}\lambda_iv_i)\\
		&=\sum_{i\neq j}^{}(\mu_i+\mu\lambda_i)v_i+\mu\lambda_jv_j\\
		&=0\\
		\\
		B=\{v_1,...,v_n\}\textrm{ Basis }&\Rightarrow\mu\lambda_j=0\textrm{ und }\mu_i+\mu\lambda_i=0\quad\forall i\neq j\\
		\underbracket{\lambda_j\neq 0}&\Rightarrow\mu=0\Rightarrow\mu_i+\underbrace{\mu\lambda_i}_{=0}=\mu_i=0\quad\forall i\neq j
	\end{align*}
	\item[2)] ($\star$) erzeugt $V$.\\
	\begin{align*}
		&&w&=\lambda_jv_j+\sum_{i\neq j}^{\lambda_iv_i}&&|\colon\lambda_j\textrm{, da }\lambda_j\neq 0\\
		&\Leftrightarrow &v_j&=\frac{1}{\lambda_j}w-\sum_{i\neq j}^{}\frac{\lambda_i}{\lambda_j}v_i&\\
		&\Rightarrow &v_j&\in\langle(B\setminus\{v_j\})\cup\{w\}\rangle_\R&\\
		&\Rightarrow&\langle(B\setminus\{v_j\})\cup\{w\}\rangle_\R&=\langle B\cup\{w\}\rangle_\R=V&
	\end{align*}
\end{itemize}
\subsection{Satz (Steinitz'scher Austauschsatz)}
Geg. $w_1,...,w_m\in V$ linear unabhängig, $\{v_1,...,v_n\}$ Basis von $V$.\\
Es folgt:
\begin{itemize}
	\item[a)] Aus den $n$ Vektoren $v_1,...,v_n$ kann man $n-m$ Vektoren auswählen, die mit $w_1,...,w_m$ eine Basis bilden.
	\item[b)] $m\leq n$
\end{itemize}
\subsubsection*{Beweis}
\begin{itemize}
	\item[a)] \begin{itemize}
		\item[1)] $w_1\in V\Rightarrow w_1=\sum_{i=1}^{n}\lambda_iv_i$\\
		Wären alle $\lambda_i=0$, dann wäre auch $w_1=0$. Da $\mathcal{O}\in V$ linear abhängig ist, wäre also auch $w_1,...,w_m$ linear abhängig. $\Lightning$\\
		Also: Mindestens ein $\lambda_i\neq 0$\\
		O.B.d.A. $\lambda_1\neq 0$ (sonst umnummerieren) $\overset{\textrm{1.20}}{\Rightarrow}\{w_1,v_2,...,v_n\}$ ist Basis von $V$
		\item[2)]  $w_2\in V\Rightarrow \mu_1w_1+\sum_{i=2}^{n}\mu_iv_i$\\
		Wären alle $\mu_2,...,\mu_n=0$, so wäre $w_2=\mu_1w_1$, also auch $w_1,w_2$ linear abhängig. $\Lightning$, da $\{w_1,...,w_m\}$ linear unabhängig.\\
		$\Rightarrow$ Mindestens ein $\mu_i\neq 0,\quad i\in\{2,...,n\}$\\
		O.B.d.A. $\mu_2\neq 0\overset{\textrm{1.20}}{\Rightarrow}\{w_1,w_2,v_3,...,v_n\}$ Basis von $V$
	\end{itemize}\hfill$\square$
	\item[b)] $\rightarrow$ Übung
\end{itemize}
\subsection{Korollar}
$V$ endlich erzeugter $\R$-VR
\begin{itemize}
	\item[i)] Je zwei Basen von $V$ enthalten gleich viele Elemente.
	\item[ii)] Basisergänzungssatz\\
	Jede linear unabhängige Teilmenge von $V$ lässt sich zu einer Basis von $V$ ergänzen.
\end{itemize}
\subsubsection*{Beweis}
\begin{itemize}
	\item[i)] $B,\tilde{B}$ Basen\\
	$B$ linear unabhängig $\overset{\textrm{1.22b)}}{\Rightarrow}|B|\leq|\tilde{B}|$\\
	$\tilde{B}$ linear unabhängig $\overset{\textrm{1.22b)}}{\Rightarrow}|\tilde{B}|\leq|B|$\\
	$\Rightarrow|B|=|\tilde{B}|$
	\item[ii)] Wähle beliebige Basis von $V$ und tausche aus(1.22a)).
\end{itemize}
\subsection{Satz}
$V$ endlich erzeugter $\R$-VR, $B\subseteq V$.\\
Dann sind äquivalent:
\begin{itemize}
	\item[i)] $B$ ist Basis
	\item[ii)] $B$ ist maximale linear unabhängige Menge in $V$
	\item[iii)] $B$ ist minimales Erzeugendensystem
\end{itemize}
\subsubsection*{Beweis}
\begin{itemize}
	\item[i)$\Rightarrow$ii)] Wegen 1.23 (linear unabhängige Menge zu Basis ergänzen, alle Basen gleich groß)
	\item[ii)$\Rightarrow$i)] (Bzw. $\neg$i)$\Rightarrow\neg$ii).) $B$ keine Basis, $B$ linear unabhängig\\
	 $\Rightarrow\langle B\rangle_\R\subsetneq V\Rightarrow\exists v\in V\setminus\langle B\rangle_\R\colon B\cup\{v\}$ linear unabhängig
	\item[i)$\Rightarrow$iii)] Satz 1.17 
\end{itemize}\hfill$\square$
\end{document}
